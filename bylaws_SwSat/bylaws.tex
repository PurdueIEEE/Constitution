% Preamble
\documentclass[12pt]{constitution}
\usepackage{mathpazo, graphicx}
\graphicspath{{figures/}}
%\usepackage[draft]{todonotes} % Show comments in PDF
\usepackage[disable]{todonotes} % Hide comments from PDF

% Metadata
\title{Software Saturdays Committee Bylaws}
\author{Event Committee of Purdue IEEE Student Branch}
\date{}
\newcommand{\datetermstart}{June 1\textsuperscript{st}} % Place term start date here
\newcommand{\dateannualreportsubmit}{May 1\textsuperscript{st}} % Place IEEE Annual Student Report submission deadline here
\newcommand{\datefiscalstart}{August 1\textsuperscript{st}} % Place fiscal year start date here
\newcommand{\dateelectionsheld}{May 1\textsuperscript{st}} % Place last day elections may be held here


\begin{document}
% Title Option #1: Title Page and Table of Contents Separate
%\maketitle
%\setcounter{tocdepth}{0} % Change to 0 for articles only or 1 for articles and sections
%\tableofcontents
%\newpage

% Title Option #2: Title at Start of Table of Contents
\titlecontentspage
\newpage


\article{Name and Purpose}
\label{art:namepurp}

Purdue IEEE Software Saturdays (hereafter Software Saturdays) exists as an event committee under the Purdue IEEE Student Branch. The purpose of Software Saturdays is to develop and deliver a series of student-to-student educational modules that impart industry-relevant software development skills to engineering students. Each module should contain lessons, review sessions, and projects exploring a broadly-applicable software skill, delivered at regular intervals over the course of a single academic term.

Although Software Saturdays and Purdue IEEE Computer Society (hereafter Computer Society) may have a similar focus, Software Saturdays differs from Computer Society in its objectives and methods of participation. Software Saturdays teaches software development frameworks to novice students. All Software Saturdays committee members are required to follow the rules established in these bylaws in addition to the Constitution of IEEE.

\article{Membership}
\label{art:member}

The members of Purdue IEEE Software Saturdays are divided into three categories: attendees, volunteer mentors, and employed mentors. Attendees consist of all people who attend any Software Saturdays lecture or review session with the intention to learn from the prepared educational module content. Volunteer mentors are the Purdue University students who prepare or deliver the instructional content in an educational module or administer the committee operations without financial compensation. Employed mentors are the students or staff employed by the Purdue University College of Engineering to contribute to Software Saturdays.

No restriction may be placed on attendees in regards to membership in Purdue IEEE Student Branch or payment of fees. Attendees who complete all requirements of an educational module may receive additional recognition for their efforts. Volunteer mentors must be members of Purdue IEEE Student Branch in good standing who obtain the approval of the Software Saturdays Chair. Volunteer mentors may be added or dismissed at any time by the Software Saturdays Chair. Employed mentors have employment through Purdue University; thus, their employment status can only be changed after consultation with a representative of the Purdue University College of Engineering. Membership in Purdue IEEE Student Branch is not a prerequisite for employed mentors, but it is still encouraged.

Software Saturdays may appoint a Faculty Advisor that is part of the faculty of Purdue University in order to supervise the direction of Software Saturdays. Other overseers may be appointed by the Purdue University College of Engineering. A Faculty Advisor and all overseers are treated as non-voting employed mentors.

\article{Leadership}
\label{art:leader}

The Purdue IEEE Software Saturdays leadership shall consist of a Software Saturdays Chair, an optional Software Saturdays Vice Chair, all Software Saturdays employed mentors, and all Software Saturdays volunteer mentors. The Software Saturdays Chair alone shall exercise the ability to vote within the Executive Committee of Purdue IEEE Student Branch. He or she shall preside over all functions of Software Saturdays and transact all business necessary to address the administrative needs of the event committee. He or she shall organize the most topical software development skills into a cohesive educational module and has the ability to delegate tasks as necessary to achieve the purposes of the event committee. The Software Saturdays Vice Chair serves as the junior executive leader of Software Saturdays and shall support the Software Saturdays Chair in the execution of his or her duties.

Volunteer mentors are expected to have at least a limited working proficiency of software programming and the software development skills needed for the current or upcoming educational module. They are expected to develop lessons for the curriculum, proofread materials, assist attendees during lessons, lead review sessions, and grade projects. While employed mentors may be assigned to any task, they typically have the same set of duties as volunteer mentors. It is incumbent upon each employed mentor to follow the employee responsibilities of Purdue University and all relevant labor laws of the State of Indiana including but not limited to reporting work hours.

The Software Saturdays Chair shall strive to serve the number of attendees as agreed upon by the Purdue University College of Engineering while maintaining the quality of instruction and an overall student-to-mentor ratio of 10:1 or less. Decisions to accept new volunteer mentors or employed mentors should be in service of this goal. The Software Saturdays Chair is responsible for informing the Faculty Advisor and IEEE President of new developments and delivering needed reports of activities to the Purdue University College of Engineering.


\article{Procedures}
\label{art:procedure}

The Executive Committee of Purdue IEEE Student Branch and the Purdue University College of Engineering jointly host Purdue IEEE Software Saturdays. As the leader who considers the holistic impact on attendees, the Software Saturdays Chair shall have the authority to make decisions on behalf of the event committee in service of the goals of the hosts. However, volunteer mentors and employed mentors may petition the Software Saturdays Chair through the medium decided by the Software Saturdays Chair for legislative or administrative changes. A simple majority of the voting mentors shall invite the IEEE President to determine what the decision shall be, potentially subject to assent from the Purdue University College of Engineering.

The Software Saturdays Chair has the responsibility for taking attendance of all the attendees at every lesson and every review session, though this task may be delegated to a volunteer mentor or employed mentor. Attendees are encouraged to submit feedback about any Software Saturdays function to the Software Saturdays Chair. Should attendees desire an alternative point of communication regarding Software Saturdays, then the IEEE President or a representative from the Purdue University College of Engineering should be contacted.

The Software Saturdays Chair shall have the power to decide access to committee resources in a manner consistent with these bylaws and charters with the Purdue University College of Engineering. Additionally, the Software Saturdays Chair shall have full control over the budgeting of all expenses and approval of all purchases. If all funding is controlled through the Purdue University College of Engineering, then those regulations control all financial activity. Otherwise, the Software Saturdays Chair is responsible for delivering the Software Saturdays budget for the academic year to the IEEE Treasurer within the first week of the academic year in accordance with the Constitution of IEEE. An updated budget for the academic year shall be submitted to the IEEE Treasurer no later than January 31\textsuperscript{st}. In such case, the IEEE Treasurer shall only reimburse purchases made with approval from the Software Saturdays Chair.


\article{Changes to Leadership}
\label{art:change}

The Software Saturdays Chair shall be appointed jointly by the IEEE President and a representative from the Purdue University College of Engineering. The Software Saturdays Chair should be a volunteer mentor or employed mentor for the upcoming academic year who maintains active membership in Purdue IEEE Student Branch. The appointment should occur at least once per academic year at least one day prior to the elections of the Purdue IEEE Student Branch. The term of office starts on \datetermstart{}.

A mentor of Purdue IEEE Software Saturdays may resign his or her role at any time. The Software Saturdays Chair and all employed mentors must receive assent from the Purdue University College of Engineering before leaving or starting their roles. It is recommended that no one be considered for an employed mentor role until he or she has served through at least one full educational module as a volunteer mentor. The Software Saturdays Chair may change the composition of volunteer mentors at any time and may also decide which mentor holds the role of Software Saturdays Vice Chair, if applicable.

The Software Saturdays Chair should be occupied at all times. In the event of a vacancy of the Software Saturdays Chair prior to the appointment, the Software Saturdays Vice Chair shall immediately assume the role of Software Saturdays Chair and work with the IEEE President and a representative of the Purdue University College of Engineering to ensure the success of the educational module underway. Should both the Software Saturdays Chair and Software Saturdays Vice Chair simultaneously be unoccupied, then the IEEE President shall work with the remaining leadership to ensure the continuity of operations.


\article{Governing Documents}
\label{art:govern}

These bylaws along with any charter accepted by the Purdue University College of Engineering serve as the governing documents for Purdue IEEE Software Saturdays. Any amendments to these bylaws shall be proposed by the Software Saturdays Chair. The bylaws shall pass with a simple majority vote of the Software Saturdays leadership. There is no minimum on mentors present for a quorum. Proxy voting is disallowed. Passed bylaws shall take effect when recognized by the Executive Committee of Purdue IEEE Student Branch as described in the Constitution of IEEE. A new charter with the Purdue University College of Engineering is needed for changes to the charter and must include the Software Saturdays Chair, IEEE President, and a Faculty Advisor at minimum in the preparation and signatures.


\article{Authority of IEEE}
\label{art:authority}

Purdue IEEE Software Saturdays acknowledges that Purdue IEEE Student Branch shares authority over its goals with the Purdue University College of Engineering. As such, Software Saturdays shall operate plainly under the name of its hosts. The IEEE President shall guide the Software Saturdays Chair as a representative of the Executive Committee of Purdue IEEE Student Branch.

In the event that the charter with the Purdue University College of Engineering falls out of effect with no succeeding charter, then the IEEE President shall assume control over all restructuring efforts for Software Saturdays. Potential options for restructuring include continuing as an event committee solely hosted by Purdue IEEE Student Branch, reorganization into a different entity within Purdue IEEE Student Branch, or disbanding while abandoning the original purpose. After restructuring, all existing Software Saturdays bylaws will lose effect.


\vspace{12pt}
\hrule

\textit{Effective: August 16\textsuperscript{th}, 2020}


\setcounter{tocdepth}{1}
% \listoftodos % Comment out when not editing


\end{document}