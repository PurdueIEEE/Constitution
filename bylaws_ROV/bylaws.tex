% Preamble
\documentclass[12pt]{constitution}
\usepackage{mathpazo, graphicx}
\graphicspath{{figures/}}
%\usepackage[draft]{todonotes} % Show comments in PDF
\usepackage[disable]{todonotes} % Hide comments from PDF

% Metadata
\title{Remotely Operated underwater Vehicle (ROV) Team Bylaws}
\author{Technical Committee of Purdue IEEE Student Branch}
\date{}
\newcommand{\datetermstart}{June 1\textsuperscript{st}} % Place term start date here
\newcommand{\dateannualreportsubmit}{May 1\textsuperscript{st}} % Place IEEE Annual Student Report submission deadline here
\newcommand{\datefiscalstart}{August 1\textsuperscript{st}} % Place fiscal year start date here
\newcommand{\dateelectionsheld}{May 1\textsuperscript{st}} % Place last day elections may be held here


\begin{document}
% Title Option #1: Title Page and Table of Contents Separate
%\maketitle
%\setcounter{tocdepth}{0} % Change to 0 for articles only or 1 for articles and sections
%\tableofcontents
%\newpage

% Title Option #2: Title at Start of Table of Contents
\titlecontentspage
\newpage


\article*{Preamble}
\label{art:preamble}

The Purdue IEEE Remotely Operated underwater Vehicle (ROV) Team exists as a committee under the Purdue IEEE Student Branch. The mission of the ROV team is to foster technical and professional skills of its members by designing, constructing, and testing an innovative underwater vehicle to compete in the MATE Center International ROV Competition. All ROV team members are required to follow the rules established in these bylaws in addition to the Constitution of IEEE.


\article{Structure and Membership}
\label{art:structmem}

The ROV team, at minimum, shall be composed of a captain (who shall be IEEE technical committee chair and vote within the Executive Committee of Purdue IEEE Student Branch), electrical team lead, mechanical team lead, and software team lead. The captain appoints and assigns work to the technical team leads named above to ensure smooth operation of the technical teams and timely completion of all tasks.

The captain has the authority to delegate his or her responsibilities to a vice captain should the captain deem it necessary. Duties that must remain with the captain and not the vice captain nor any other ROV team member include final budget decisions, unlimited reimbursement approval power, changes to composition or organization of the team roster, and selection of members to attend the MATE Center International ROV Competition with team funds. If the captain deems it necessary, the captain may introduce project groups as a second form of organization beyond technical teams. The captain has the ability to add or subtract or change membership within project groups at any time. Each project group has a project group head appointed by the captain with the duties of keeping the project on track, reporting regularly to the captain, and mediating minor disputes between members on that project. Other, non-team lead positions may be appointed and removed at the captain's leisure.

Attendance at a majority of team meetings and continuous participation with meaningful contributions throughout the design, construction, testing, and documentation phases of the ROV are expected of all members. Additionally, attendance at a majority of technical team meetings to which the respective member belongs is expected. Infrequent attendance, low number or quality of contributions made, disruptive behavior, or lack of IEEE dues payment may be cited as reasons to not be recognized as a member, selected to represent the team at outreach events, or allowed to attend competitions.


\article{Decisions}
\label{art:decide}

While decisions should usually be made with the advice and consent of the team, the captain shall have the ability to make or delegate all final team decisions. However, if there is a disagreement, a unanimous vote of technical team leads shall result in a vote over the decision being taken by the whole team. A majority of voting team members will override any captain's decision.

The captain shall have the power to decide access to team resources in a manner consistent with these bylaws. Additionally, the captain will be responsible for the approval of team purchases. This process may be delegated by the captain as needed. Only with approval will a purchase be eligible for reimbursement by the IEEE Treasurer. The captain is responsible for delivering the ROV budget for the season to the IEEE Treasurer within the first week of the academic year in accordance with the Constitution of IEEE.

The captain shall ensure that all members are trained in their respective technical fields in alignment with the goals of IEEE Learning Committee at the start of the academic year.

While members of the team may be removed or demoted by the captain, the captain may lose his or her position only in the same voting procedure listed above in this article.

In the event of a temporary captain vacancy less than a month, the captain may appoint a temporary replacement with full powers of the captain in his or her absence. In all other circumstances, an election for an acting captain with full powers of captain should be held as soon as reasonably possible in the way defined in Article \ref{art:elect}. The acting captain shall retain the full powers of captain until the return of the original captain, at which point, the full powers of captain return to the original captain.


\article{Elections}
\label{art:elect}

Before the beginning of each new season, there shall be a vote for captain. A season shall consist of no more than one international competition. The election for captain should occur at least a day before IEEE elections; however, the current captain can request a deferral of the vote to the elected officers of IEEE. The newly elected captain should participate in the IEEE Executive Committee for the class taking office after the season ends. The current captain shall remain in power for the remainder of the season.

A captain must have been a member of the team for at least four months and satisfy all other eligibility requirements imposed by the Constitution of IEEE. All current members of the team at the time of the election shall have one vote. Eligibility for voting shall be decided by the captain with regards to the aforementioned bylaws. A majority of voting team members shall elect the captain. In the event no majority exists, the candidate with the fewest votes shall be removed from the ballot and another vote will occur. If there is still no majority among the voting members when there are only two members left, the previous captain shall decide the new captain.


\article{Bylaws}
\label{art:bylaw}

Any amendments to the bylaws shall be proposed by any member and be brought to a vote only upon approval by the captain. The bylaws shall pass with a majority of voting team members or by a unanimous team lead vote.

\vspace{12pt}
\hrule

\textit{Effective: October 12\textsuperscript{th}, 2017}


\setcounter{tocdepth}{1}
%\listoftodos % Comment out when not editing


\end{document}
