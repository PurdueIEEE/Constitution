% Preamble
\documentclass[12pt]{constitution}
\usepackage{mathpazo, graphicx}
\usepackage{soul}
\graphicspath{{figures/}}
%\usepackage[draft]{todonotes} % Show comments in PDF
\usepackage[disable]{todonotes} % Hide comments from PDF

% Metadata
\title{Constitution of IEEE}
\author{Purdue University -- West Lafayette Student Branch}
\date{}
\newcommand{\datetermstart}{June 1\textsuperscript{st}} % Place term start date here
\newcommand{\dateannualplansubmit}{February 1\textsuperscript{st}} % Place IEEE Student Branch Annual Plan submission deadline here
\newcommand{\datefiscalstart}{August 1\textsuperscript{st}} % Place fiscal year start date here
\newcommand{\dateelectionsheld}{May 1\textsuperscript{st}} % Place last day elections may be held here


\begin{document}
% Title Option #1: Title Page and Table of Contents Separate
%\maketitle
%\setcounter{tocdepth}{0} % Change to 0 for articles only or 1 for articles and sections
%\tableofcontents
%\newpage

% Title Option #2: Title at Start of Table of Contents
\titlecontentspage
\newpage


\article*{Preamble}
\label{art:preamble}

We, the students of IEEE, organize this Purdue University-West Lafayette Student Branch of IEEE to facilitate member growth by pursuing technical interests, developing as professionals, learning diverse skills and ideas, and building social relationships between our members and others on this campus and in the community. 


\article{Name}
\label{art:name}

The name of this organization is IEEE. The pronunciation is ``Eye-triple-E``. When necessary to differentiate this student organization from the international professional association of the same name, the name Purdue IEEE Student Branch is used.


\article{Purpose}
\label{art:purpose}

\textit{The purposes of this organization are:}

\section{Administrative Goal}
\label{sec:purp_admin}
To serve as a Student Branch of IEEE at the Purdue University--West Lafayette campus and operate in accordance with the Constitution and Bylaws of the IEEE.

\section{Learning Goal}
\label{sec:purp_learn}
To disseminate knowledge of the theory and practice of all branches of engineering and related arts and sciences with specific emphasis on aspects of electrical and computer engineering, electronics, and communications.

\section{Technical Goal}
\label{sec:purp_tech}
To put the aforementioned knowledge and theory into practice, specifically through technical committee activities.  

\section{Industrial Relations Goal}
\label{sec:purp_prof}
To enhance the professional development of its members through events and programming useful to the growth of students and through cooperation with our partners in industry.

\section{Social Goal}
\label{sec:purp_soc}
To encourage personal and professional relationships among our members, other student organizations, and the global technical community through IEEE.

\section{Growth and Engagement Goal}
\label{sec:purp_out}
To advocate for the strength of an education from Purdue University and to encourage younger students to pursue careers in science, technology, engineering, and mathematics (STEM).


\article{Membership}
\label{art:members}

\textit{Definition of Membership:}

\section{Inclusiveness Clause}
\label{sec:mem_inclusive}
\begin{itemize}
    \item Membership and participation are free from discrimination on the basis of race, religion, color, sex, age, national origin or ancestry, genetic information, marital status, parental status, sexual orientation, gender identity and expression, disability, or status as a veteran.
\end{itemize}

\section{Membership Categories}
\label{sec:mem_categ}
\begin{itemize}
    \item The membership shall consist of
    \begin{enumerate}
        \item Undergraduate and Graduate Student members of IEEE at Purdue University (registered for either physical instruction or online instruction) in good standing,
        \item the Branch Counselor, the Primary Advisor, Chapter Faculty Advisors, and Secondary Advisors (faculty or staff) at Purdue University, and
        \item honorary members recognized by the Executive Committee.
    \end{enumerate}
\end{itemize}

\section{Advisors}
\label{sec:mem_advis}
\begin{itemize}
    \item In compliance with IEEE Student Branch guidelines, the IEEE Student Branch shall select a Branch Counselor from the faculty of Purdue University who holds active membership in IEEE.
    \item This individual may also be selected as the Primary Advisor or as a Secondary Advisor of the student organization as mandated by Purdue University.
\end{itemize}

\section{Voting Members}
\label{sec:mem_vot}
\begin{itemize}
    \item Current undergraduate and graduate student members are entitled to full rights and voting privileges as members of the Student Branch while faculty or staff at Purdue University are entitled to the same rights except to hold office or vote.
\end{itemize}

\section{Joining}
\label{sec:mem_join}
\begin{itemize}
    \item All current undergraduate and graduate student members may join the Purdue IEEE Student Branch.
    \item Reporting requirements of IEEE may restrict membership definitions to include students actively holding Student or Graduate Student levels of membership within the international organization.
\end{itemize}

\section{Active Membership}
\label{sec:mem_active}
\begin{itemize}
    \item Active membership shall be defined as attending three or more general assemblies, social functions, officer meetings, or committee meetings in an academic year.
    \item Good standing requires compliance with any local dues and any requirements passed by the Executive Committee.
    \item Members are eligible for a leadership role in any IEEE committee if and only if they are in good standing with the Student Branch.
\end{itemize}

\section{Disciplinary Actions}
\label{sec:mem_discip}
\begin{itemize}
    \item Any member displaying unprofessional, inappropriate, or dangerous behavior or engaging in illegal activities while representing IEEE or Purdue University may be subject to disciplinary actions.
    \item Such actions may include
    \begin{itemize}
        \item An email describing the incident and a demand to cease such behavior or activities,
        \item Review before the officers and committee chairs,
        \item Enactment of a probationary period where membership rights may be suspended upon an Executive Committee supermajority vote (ECSV) as defined in Article \ref{art:excom}, Section \ref{sec:exec_vote}, and
        \item Formal disassociation and expulsion from the Purdue IEEE Student Branch upon an ECSV.
    \end{itemize}
    \item Outside of the actions outlined above, any incident may be reported to the Office of Student Rights and Responsibilities or the Office of Professional Practice.
\end{itemize}


\article{Executive Committee}
\label{art:excom}

\section{Composition}
\label{sec:exec_comp}
\begin{itemize}
    \item The management of the affairs of the Student Branch shall be in the hands of the Executive Committee, consisting of the duly elected officers of the Student Branch, technical committee chairs, and the event committee chairs.
    \item The President shall be the Chair of the Executive Committee.
    \item The President is the \textit{ex officio} Chair of the Branch in regards to reports to IEEE. Similarly, the Vice President of Technical Committees is the \textit{ex officio} Vice-Chair of the Branch, and the Vice President of Member Involvement is the ex officio Membership Development Chair of the Branch.
    \item Technical committee chairs and event committee chairs must be selected by their committees by election time.
    \begin{itemize}
        \item Succeeding committee chairs must be students of Purdue University at the start of the next academic year and be in good standing with the Purdue IEEE Student Branch.
        \item Exceptions are listed in Article \ref{art:technical}, Section \ref{sec:tech_govern} and Article \ref{art:event}, Section \ref{sec:event_govern}.
    \end{itemize}
    \item Under most circumstances, technical committee chairs and event committee chairs shall be granted the same privileges as elected officers.
    \item The Branch Counselor, the Primary Advisor, Chapter Faculty Advisors, and Secondary Advisors are appointed by an Executive Committee majority vote (ECMV), and may sit on Executive Committee meetings as honorary members but lack the right to vote. No term limits apply to these positions.
\end{itemize}

\section{Voting Procedures}
\label{sec:exec_vote}
\begin{itemize}
    \item A quorum is defined as one greater than half of the voting Executive Committee members. A quorum is necessary for the proceedings of any Executive Committee meeting to be binding.
    \item Presence for a vote shall be defined as the ability to communicate via the indicated meeting format (e.g., in-person, video conference, etc.).
    \begin{itemize}
        \item Individual alternatives for presence are only accepted in special cases as determined by the President, Vice President of Technical Committees, or Vice President of Member Involvement granting approval.
        \item Proxy voting is strictly disallowed.
    \end{itemize}
    \item Members of the Executive Committee serving in multiple roles are limited to one vote on each matter being considered by the committee.
    \item An Executive Committee majority vote (ECMV) passes when one greater than half of the Executive Committee members present at a meeting vote in favor of a motion. Decisions of the Executive Committee will be decided by ECMV unless otherwise indicated in this constitution.
    \item An Executive Committee supermajority vote (ECSV) passes when at least two-thirds of the total Executive Committee members vote in favor of a motion.
    \item An elected officer majority vote (EOMV) passes when one greater than half of the elected officers listed in Article \ref{art:elect}, Section \ref{sec:elect_list} vote in favor of a motion.
\end{itemize}

\section{Powers and Limitations}
\label{sec:exec_pow}
\begin{itemize}
    \item The Executive Committee shall be the governing body of the Student Branch and shall transact all business it deems advisable to fulfill executive responsibilities, including those not otherwise delegated by this Constitution.
    \item The President and Treasurer must jointly approve purchases for goods or services that benefit the whole membership of IEEE valued at more than USD 100.
    \item Purchases specific to each committee made in full with committee funds lie with that committee and outside the purview of the Executive Committee.
    \item Sponsorship disputes will be decided by all voting members of the Executive Committee not recusing themselves.
    \item Normal business of technical committees that lacks a measurable impact on the organization as a whole is not subject to Executive Committee approval.
    \item Powers specific to each office are outlined in Article \ref{art:officers}. The passage of amendments in Article \ref{art:amend}, the collection of dues in Article \ref{art:dues}, and the responsibility of organizing committees in Article \ref{art:cornerstones}, Article \ref{art:technical}, and Article \ref{art:event} are detailed in the mentioned articles.
\end{itemize}

\section{Rakos Funding Distribution Act}
\label{sec:rakos_funding_distribution_act}
\begin{itemize}
    \item Any funding received from any source that is made out to Purdue IEEE Student Branch without explicitly mentioning a committee or specific entity on the account statement or through other communication to receive the funds shall adhere to the following procedures.
    \item The funding shall be split exclusively and evenly between committees or the Student Branch as a whole who have contacted the company or source for funding or sponsorship within the previous ten months successfully.
    \begin{itemize}
        \item Success in contacting a company will be determined by Executive Committee consensus. If consensus cannot be found, a final decision on a successful contact will be determined by the Primary Advisor.
    \end{itemize}
    \item If no committee has contacted the company within the previous ten months but a committee has previously received funding from the same company due to previous contact, and has contacted the company for sponsorship, successfully or not, in the past three years, they shall receive 50\% of the funding and the remainder shall be given to the Student Branch as a whole.
    \item If multiple committees are in the aforementioned situation, then 50\% of the funding shall be split evenly between these multiple committees.
    \item In all other cases, the money shall be given the Student Branch as a whole.
    \begin{itemize}
        \item The preceding point will not prohibit the Executive Committee from voting to redistribute the funding as desired; any such vote is strongly encouraged to split the funds evenly between the Cornerstones committees, technical committees, and Student Branch combined.
    \end{itemize}
\end{itemize}

\section{Departure from Office}
\label{sec:exec_depart}
\begin{itemize}
    \item In the event of the graduation or resignation of an elected officer, barring Student Branch Chapter Chairs, the departing officer may nominate a successor to be approved by ECMV of only the remaining Executive Committee members. In the case of a failed vote, the President may nominate the next successor to take office subject to a vote by the same.
    \item Technical committees and event committees must specify their procedure for filling chair vacancies in their respective technical committee bylaws or event committee bylaws. In the case that a committee vacancy is not resolved within ten days, the President may nominate a replacement subject to an ECMV.
    \item An officer or committee chair may be recalled from office when at least three members collectively request a recall vote. At least one week must elapse between the announcement of a recall vote including the manner in which it will be held and the vote itself. Recalled officers or committee chairs may not nominate successors and may additionally be subject to the disciplinary actions given in Article III, Section 7 provided that the conditions given in said section are met.
    \item One manner of successful recall vote is by ECSV in favor of removing that member. In this case, the elected officer or committee chair in question may not vote. Alternatively, a general assembly of IEEE members that includes at least two members from each technical committee not serving on the Executive Committee may recall an elected officer with a two-thirds vote.
    \item Should the current President become unable to serve without nominating a successor the Vice President of Technical Committees shall assume the Presidency.
    \begin{itemize}
        \item Should both of those offices be vacant, then the succession proceeds down to the Vice President of Member Involvement, Treasurer, Secretary, Industrial Relations Head, Growth and Engagement Head, Learning Head, and Social Head in turn.
        \item The new President may then nominate a successor to the former office he or she held and any other vacant office subject to an ECMV.
    \end{itemize}
\end{itemize}

\section{Ad Hoc Committees}
\label{sec:exec_adhoc}
\begin{itemize}
    \item Ad hoc committees may be formed or appointed by any elected officer to assist with the fulfillment of his or her constitutional duties.
    \begin{itemize}
        \item Ad hoc committees are designed to be for a specific event or group of events. Larger, continuous events should result in the formation of an event committee.
        \item Ad hoc committees are meant to delegate specific duties that might otherwise be too demanding of an elected officer's time.
    \end{itemize}
    \item Ad hoc committees may be granted additional power or have it removed on a case-by-case basis within the scope of the Executive Committee's power upon an ECMV.
    \item The chair of an ad hoc committee is expected to report to the elected officer before each Executive Committee meeting.
    \begin{itemize}
        \item If the ad hoc committee was formed under the Executive Committee as a whole, then the chair should report to the Executive Committee meetings unless directed otherwise.
	\item The chair of an ad hoc committee cannot be an elected officer.
    \end{itemize}
    \item These committees will be automatically dissolved by \datetermstart{} of that election cycle or by the Executive Committee with an ECMV.
\end{itemize}


\article{Officers and Duties}
\label{art:officers}

\section{List of Voting Executive Committee Members}
\label{sec:officer_list}
\begin{itemize}
    \item President
    \item Vice President of Technical Committees
    \item Vice President of Member Involvement
    \item Treasurer
    \item Secretary
    \item Industrial Relations Head
    \item Growth and Engagement Head
    \item Learning Head
    \item Social Head
    \item Committee Chairs
\end{itemize}

\section{Duties of the President}
\label{sec:officer_pres}
The President shall have the following duties:
\begin{itemize}
    \item Ensure adherence to the IEEE Code of Ethics.
    \item Maintain the Purdue IEEE Code of Conduct.
    \item Organize and preside at all recruitment events for the Student Branch.
    \item Set regular meeting times for the Executive Committee and preside over them.
    \item Assume responsibility for the operations of the Executive Committee and complete delinquent tasks of elected officers within the letter and spirit of this Constitution.
    \item Promote and support the activities of the committees.
    \item Oversee the activities of all event committees.
    \item Cultivate communication between members.
    \item Manage member activities and oversee the planning of general assemblies.
    \item Register the student organization with Purdue Student Activities and Organizations.
    \item Maintain necessary contact with Purdue Student Activities and Organizations and the Purdue Business Office for Student Organizations.
    \item Consult with the Primary Advisor regularly and complete necessary paperwork with him or her.
    \item Maintain good relations with other student organizations, and especially other professional organizations.
    \item Interact with IEEE Student Branch Chapters operating as independent student organizations at Purdue University for reporting purposes.
    \item Interact with the local community on behalf of IEEE.
    \item Approve documented purchases for goods or services that benefit the whole membership of IEEE valued strictly under USD 100 at his or her discretion.
    \item Ensure the transition of information and organization materials from the current class of the Executive Committee to the newly elected officers and committee chairs.
    \item Represent Purdue University-West Lafayette at meetings of the Central Indiana Section of IEEE and IEEE Region 4.
    \item Report changes in officers and complete the Student Branch Annual Plan to submit by \dateannualplansubmit{}.
    \item Secure the trust and confidence of the members of IEEE.
\end{itemize}

\section{Duties of Vice President of Technical Committees}
\label{sec:officer_vp_Tech}
The Vice President of Technical Committees shall have the following duties:
\begin{itemize}
    \item Perform the functions of the President in the latter's absence or at his or her request.
    \item Assist the President in his or her duties.
    \item Oversee the drafting, amending, adherence, and enforcement of the Constitution of IEEE.
    \item Plan and lead all general assemblies of the Student Branch.
    \item Strive to acquaint himself or herself with knowledge pertaining to the history of the organization and spread the knowledge.
    \item Respond directly to the issues raised by technical committee chairs and resolve conflicts between technical committees.
    \item Mediate unresolved disagreements occurring between multiple technical committees.
    \item Help committees develop plans to meet their monetary, space, and personnel needs.
    \item Be aware of the focus areas of each technical committee to inform prospective members of projects interesting to them.
    \item Remind technical committee members of professional, learning, and social activities.
    \item Bring attention to the advantages of international IEEE membership.
    \item Spread awareness of IEEE across campus.
    \item Work with the Secretary to gather members for outreach events.
    \item Coordinate safety and responsibility training for any IEEE workspace shared among committees.
\end{itemize}

\section{Duties of Vice President of Member Involvement}
\label{sec:officer_vp_Inv}
The Vice President of Member Involvement shall have the following duties:
\begin{itemize}
    \item Assist the Vice President of Technical Committees in drafting and amending the Constitution of IEEE.
    \item Ensure each member abides by this Constitution together with the Vice President of Technical Committees.
    \item Assist the Vice President of Technical Committees in planning general assemblies.
    \item Supervise meetings coordinating the planning activities of the Cornerstones committees.
    \item Respond directly to the issues raised by Cornerstones committee Heads and resolve  conflicts between the Cornerstones committees.
    \item Mediate unresolved disagreements occurring between multiple Cornerstones committees.
    \item Help Cornerstones committees fulfill their plans for the academic year.
    \item Encourage technical committee members to volunteer as representatives on Cornerstones committees.
    \item Collaborate with the officers of the Beta Chapter of Eta Kappa Nu (HKN) to maintain good relations and hold events for the benefit of the members of both IEEE and HKN.
\end{itemize}

\section{Duties of the Treasurer}
\label{sec:officer_treas}
The Treasurer shall have the following duties:
\begin{itemize}
    \item Oversee the finances of IEEE and its committees.
    \item Keep accounts, deposit the organization's funds, and make expenditures in a manner approved by the Business Office for Student Organizations.
    \item Maintain necessary contact with Purdue Student Activities and Organizations and the Purdue Business Office for Student Organizations.
    \item Approve expenditures and process reimbursements for every committee provided the committee has sufficient funds, adheres to this Constitution, and shows a relationship between the purchase and furtherance of the goals of that committee.
    \item Complete reimbursements and produce committee spending reports in a timely fashion and keep clear, accurate records to be passed to the next Treasurer indefinitely under penalty of dismissal for excessive tardiness.
    \item Advance a budgetary plan for each academic year within the first month of the academic year under penalty of dismissal from office.
    \item Learn the tools necessary for the completion of these tasks.
    \item Deliver a statement of organization finances at each Executive Committee meeting unless a motion to forego that proceeding passes by ECMV.
    \item Collect dues assigned by the process in Article VIII and distribute IEEE merchandise and trinkets.
    \item Mark the beginning of financial records for the current fiscal year on \datefiscalstart{}.
    \item All financial business for the current fiscal year must be completed before \datefiscalstart{}.
    \item Prepare a complete financial statement of the previous calendar year as part of the Student Branch Annual Plan for submission by \dateannualplansubmit{}.
    \item Notify the Executive Committee of unreceived funding and unclearly labeled funding given to the Student Branch on a regular interval of 60 days or less during the fiscal year.
\end{itemize}

\section{Duties of the Secretary}
\label{sec:officer_sec}
The Secretary shall have the following duties:
\begin{itemize}
    \item Keep a record of all activities of the Student Branch as a whole, including the names of members in attendance at the meetings.
    \item Write minutes of all Executive Committee meetings and general assemblies and deliver them to the succeeding Secretary.
    \item Record the language of every item to be voted upon and the numeric tally of the members voting in a given manner. Votes shall be tallied by the highest-ranking member in order of succession as detailed in Article \ref{art:excom}, Section \ref{sec:exec_depart}. \todo[backgroundcolor=red!25]{MRH: It is weird to have vote tallying mentioned in this place.}
    \item Choose to enforce order at all meetings at the request of the President or Vice President.
    \item Maintain the membership roster including names, contact information, academic program information, and committee affiliations.
    \item Maintain the calendar, websites, mailing list, and other Internet media belonging to IEEE with current information and news with assistance from designated individuals which may be appointed as needed by the Executive Committee.
    \item Answer questions asked by prospective members about organization membership.
    \item Report all organization activities to IEEE Headquarters, together with any special reports required by IEEE Headquarters.
    \item Forward efforts to recognize members for their technical accomplishments and service inside and outside the organization.
    \item Apply for the IEEE Regional Exemplary Student Branch Award before the deadline each year unless determined otherwise by an ECSV.
    \item Carry on all other communications necessary to the activities of the Branch. 
    \item Review the Student Branch Annual Plan prior to the submission by the President.
\end{itemize}

\section{Duties of the Industrial Relations Head}
\label{sec:officer_prof}
The Industrial Relations Head shall have the following duties:
\begin{itemize}
    \item Serve as committee chair of the Industrial Relations Committee.
    \item Organize industry nights for sponsors and serve as a point of contact for them.
    \item Focus on the acquisition and retention of sponsors in addition to the explanation and recommendation of them to students.
    \item Assist with the coordination of sponsorship acquisition amongst technical committees. \todo[backgroundcolor=red!25]{MRH: Mentoring is fine, but might remove the actual acquisition for technical committees as a stated aim.}
    \item Host career building and soft skills events to further professional development.
    \item Advance a corporate engagement plan for each academic year within the first month of the academic year under penalty of dismissal from office.
    \item Converse with professors from the variety of disciplines the body of members study and invite them to hold events with IEEE members.
    \item Assemble directories of company contacts, professors, and company information for future Professional Heads.
	\item Lead sponsorship efforts for the benefit of the entire Student Branch.
    \item Shall maintain a record of the companies, organizations, and departments each Cornerstones committee,technical committee, and event committee has contacted for sponsorship.
\end{itemize}


\section{Duties of the Growth and Engagement Head}
\label{sec:officer_growth}
The Growth and Engagement Head shall have the following duties:
\begin{itemize}
    \item Serve as committee chair of the Growth and Engagement Committee.
    \item Advance a member engagement plan for each academic year within the first month of the academic year under penalty of dismissal from office.
    \item Interact with organization alumni and maintain a contact list of them.
    \item Lead efforts to promote and retain membership in IEEE among first-year students.
    \item Organize outreach events for Student Branch.
    \begin{itemize}
        \item Interact with the local community, particularly to encourage grade school and high school students to pursue STEM and IEEE.
        \item Organize outreach events with IEEE section and region members.
    \end{itemize}    
    \item Involve underrepresented groups in IEEE activities.
\end{itemize}

\section{Duties of the Learning Head}
\label{sec:officer_learn}
The Learning Head shall have the following duties:
\begin{itemize}
    \item Serve as committee chair of the Learning Committee.
    \item Concentrate on the development of the technical competencies of new members.
    \item Promote teamwork and soft skills in all possible aspects for maximum employability.
    \item Provide the appropriate level of challenge that each member seeks in extracurricular involvement.
    \item Organizing events to teach members about the topics in Article \ref{art:purpose}, Section \ref{sec:purp_learn}.
    \item Advance a learning event plan for each academic year within the first month of the academic year under penalty of dismissal from office.
    \item Seek out the technical leaders and experts within IEEE to lead workshops open to all members.
    \item Establish appropriate incentives and rewards for technical leader involvement.
    \item Cater to the breadth of academic and career interests that IEEE members display, centralize the common technical committee training needs, and teach topics beyond them.
    \item Ensure members have easy access to high-quality instructional materials after workshops in organized repositories.
    \item Guide the development of lesson plans, notes and learning material, recorded guides, instructor directions, skills evaluation procedures, and documentation for continued offerings of learning events under future Learning Heads.
    \item Create study groups for courses and mentorship programs as necessary.
    \item Work with the School of Electrical and Computer Engineering to maintain the legitimacy and sanction of learning efforts.
\end{itemize}

\section{Duties of the Social Head}
\label{sec:officer_soc}
The Social Head shall have the following duties:
\begin{itemize}
    \item Serve as committee chair of the Social Committee.
    \item Bear the responsibility of organizing social functions for the benefit and enjoyment of the members.
    \item Advance a member engagement plan for each academic year within the first month of the academic year under penalty of dismissal from office.
    \item Encourage attendance at organization events through advertising well in advance of each event.
    \item Produce social media posts, email messages, and the like that contain Cornerstones committee and important IEEE updates to share with all members.
    \item Save notes of successful plans and implementations of events for the next Social Head.
    \item Monitor the contentment of individual members and take steps to address their concerns when possible.
    \item Promote inclusiveness within the organization and each technical committee.
    \item Provide ways for members not belonging to technical committees to engage with IEEE.
    \item Facilitate the growth of personal friendships between IEEE members.
\end{itemize}


\article{Elections}
\label{art:elect}

\section{Elected Offices}
\label{sec:elect_list}
The following Executive Committee offices shall be elected by voting IEEE members:
\begin{itemize}
    \item President
    \item Vice President of Technical Committees
    \item Vice President of Member Involvement
    \item Treasurer
    \item Secretary
    \item Industrial Relations Head
    \item Growth and Engagement Head
    \item Learning Head
    \item Social Head
\end{itemize}

\section{Eligibility}
\label{sec:elect_elig}
To be eligible for office:
\begin{itemize}
    \item The candidate must be a student at Purdue University during elections and at the start of the next academic year.
    \item The candidate must be in good standing with the Purdue IEEE Student Branch.
    \item The candidate must not hold another newly elected office.
    \item The President, Vice President of Technical Committees, Vice President of Member Involvement, and Treasurer must forfeit any committee chair positions for the duration they hold any one of these offices.
    \item Officers should enroll as international IEEE members for the term being served.
    \item A candidate will be removed from the ballot for a subsequent office prior to voting if he or she is elected to another office earlier during elections unless the candidate vacates his or her prior position.    
\end{itemize}

\section{Term}
\label{sec:elect_term}
\begin{itemize}
    \item The term of office shall be for one year, starting on \datetermstart{}.
    \item Elections must occur before \dateelectionsheld{}.
\end{itemize}

\section{Nominations}
\label{sec:elect_nom}
\begin{itemize}
    \item The student IEEE members shall give nominees for each office to the Executive Committee in a period from twenty-eight days to three days before election time.
    \item Self-nominations are allowed.
    \item To be eligible for a vote, nominated individuals must give a written acceptance of their nomination to the President (for each of the offices to which he or she has been nominated) prior to three days before election time.
    \item The President must forward all eligible candidates for each office to the teller decided according to Article \ref{art:elect}, Section \ref{sec:elect_teller} before election time.
\end{itemize}

\section{Voting Process}
\label{sec:elect_vote}
\begin{itemize}
    \item The members shall select the elected officers by a two-round voting process.
    \begin{itemize}
        \item For each office, the voting members may cast a single vote for one of the candidates.
        \item The candidate with a majority of the votes wins.
        \item Should no candidate receive a majority of the votes, a second round of voting will begin with only the top two candidates on the ballot.
    \end{itemize}
    \item The order of voting is that mentioned in Article \ref{art:elect}, Section \ref{sec:elect_list}.
    \item Non-officer elected positions shall be decided after the elections for the officers.
    \item The teller must accurately tally and record the votes for each candidate for each office, announce the results, and deliver the documentation of the tallies to the incoming Secretary.
\end{itemize}

\section{Tie Breaking}
\label{sec:elect_tie}
\begin{itemize}
    \item In the event the members at elections yield a tie for the top candidates for an office, the current elected officers and committee chairs shall decide that election by the method described in Article \ref{art:elect}, Section \ref{sec:elect_vote}.
    \item Should the current elected officers and committee chairs produce a further tie, then the winning candidate will be decided by sortition.
\end{itemize}

\section{Teller and Vote Collection}
\label{sec:elect_teller}
\begin{itemize}
    \item The Primary Advisor, Branch Counselor, or secondary advisor shall be the teller of the election, tallying the votes of present students.
    \item In their absence, the President will decide a teller.
    \item Presence for election voting shall be defined by the ability to vote in person or remotely.
    \item Remote votes will be granted by the President only for extenuating circumstances and must be delivered in a signed and dated format to the teller on the day of elections prior to the start of voting.
    \item Candidates must be listed in preferential order separated by office so that the most preferential, eligible candidate for the office shall receive each remote student's vote for that office.
\end{itemize}


\article{Amendments and House Rules}
\label{art:amend}

\section{Amendment Process}
\label{sec:amend_process}
\begin{itemize}
    \item The Executive Committee of IEEE may propose amendments to this Constitution during any of their meetings.
    \item Amendments pass only upon an ECSV.
    \item Corrections of typographical (limited to page formatting, spelling, capitalization, and spacing) errors in this Constitution are not subject to approval through ECSV.
    \item Corrections can only be made provided that adequate notice of them is given at the subsequent meeting of the Executive Committee and records are given to all officers and no officer dissents.
    \item The Vice President of Technical Committees and Vice President of Member Involvement may independently correct typographical errors to this Constitution.
\end{itemize}

\section{Amendment Approval}
\label{sec:amend_approve}
\begin{itemize}
    \item All amendments to the Constitution and Bylaws are subject to the approval of the Office of Student Activities and Organizations.
    \item Amendments may not take effect until after they have been approved.
\end{itemize}

\section{House Rules}
\label{sec:house_rules}
\begin{itemize}
\item Any member of the Executive Committee of IEEE may propose a house rule that does not conflict with the Constitution of IEEE to the President during any meeting.
\item The President has control over which house rules are called to a vote.
\item House rules may not change who has the power to vote, Executive Committee powers and limitations, Cornerstones committee operations, technical committee operations, or event committee operations.
\item House rules pass or are removed upon an ECSV.
\item All house rules take effect immediately unless otherwise stated.
\item All house rules expire with a change in the office of President.
\item The Secretary will make all house rules in effect for the Executive Committee and the date they were passed available in a single document for every member of the IEEE Executive Committee to view.
\end{itemize}

\section{Committee Bylaws}
\label{sec:amend_techbylaw}
\begin{itemize}
    \item Each technical committee chair or event committee chair must give the Executive Committee written affirmation to abide by the bylaws for his or her committee within the first four weeks of the academic year.
    \item All technical committees and event committees must write and pass bylaws specific to them that chronicle their purpose, internal structure, member rights, committee chair succession, and established procedures.
    \item Each technical committee chair or event committee chair must deliver an updated written copy of the bylaws to his or her committee and the Executive Committee within ten days of any modifications.
    \item While most technical committee business is conducted internal to each committee, outstanding issues shall be resolved by the Vice President. In the event the issues remain or grievances against the decision exist, final power shall devolve to the Executive Committee without restrictions beyond this Constitution.
    \item While most event committee business is conducted internal to each committee, outstanding issues shall be resolved by the President. In the event the issues remain or grievances against the decision exist, the Executive Committee excluding the President shall have the power to enact any further remedies by ECMV excluding the President.
    \item The Executive Committee will review technical committee bylaws and event committee bylaws for guidance but are not bound to their procedures in these situations.
\end{itemize}


\article{Dues}
\label{art:dues}

\section{Instituting Dues}
\label{sec:dues_institute}
\begin{itemize}
    \item Dues must be collected from members listed on the roster within four weeks of the callout, registered committee information session, or addition to the roster, whichever is latest. Committee members who have not paid dues will be unable to partake in any IEEE activities hosted by any committee until they pay dues.
    \item The Executive Committee shall have the power to levy special assessments for the present time up until the end of the term of office upon endorsement via an ECSV.
    \item Upon expiration of the assessment period, a continuance or new assessment is required to continue collecting dues following an ECSV.
    \item It is recommended but not required that Graduate Student Members with active international IEEE membership be exempted from local dues to encourage their inclusion and guidance in Student Branch activities.
\end{itemize}

\section{Nonpayment}
\label{sec:dues_nonpay}
\begin{itemize}
    \item The Treasurer is responsible for the collection of dues and recording the payment in the roster kept by the Secretary.
    \item Nonpayment of local dues by the date determined by ECMV may result in suspension of membership in the local Student Branch but cannot result in suspension of membership at the international professional organization level.
\end{itemize}


\article{Cornerstones Committees}
\label{art:cornerstones}

\section{Designation}
\label{sec:corner_desig}
\begin{itemize}
    \item The Cornerstones committees shall consist of the Industrial Relations Committee, Growth and Engagement Committee, Learning Committee, and Social Committee.
    \item The Industrial Relations Head, Growth and Engagement Head, Learning Head, and Social Head shall have exclusive rights to the word ``Head'' in their titles to reflect the special nature of their selection and duties.
\end{itemize}

\section{Membership}
\label{sec:corner_auto}
\begin{itemize}
    \item Each Cornerstones committee is responsible for recruiting membership from the Student Branch as a whole to help plan all events.
    \item All members of IEEE as defined in Article \ref{art:members} are automatically entitled to the benefits and programs of each.
\end{itemize}

\section{Representatives and Responsibilities}
\label{sec:corner_represent}
\todo[inline,backgroundcolor=red!25]{MRH: It might be time to add that a technical committee chair may \textbf{not} serve as a representative.}
\begin{itemize}
    \item The representatives on each Cornerstones committee shall assist each elected Head in his or her duties as listed in Article \ref{art:officers}, Sections \ref{sec:officer_prof}-\ref{sec:officer_soc}.
    \begin{itemize}
        \item The representatives within each Cornerstones committee serve at the leisure of the elected Head.
        \item The Head should appoint members within his or her Cornerstones committee as representatives to inform all members within IEEE of their activities, especially those in technical committees and event committees.
        \item The Head may choose to be a representative. It is recommended that the Head make others members representatives.
    \end{itemize}
    \item All members of Purdue IEEE Student Branch are welcome to participate in Cornerstones committee planning activities.
    \item Each elected Head is required to communicate all upcoming events and meetings of their Cornerstones committee with the Vice President of Member Involvement, attend regular meetings with the Vice President of Member Involvement unless excused for the occasion, and report the outcomes of each event as requested by the Vice President of Member Involvement.
\end{itemize}

\section{Finances}
\label{sec:corner_finance}
\begin{itemize}
    \item Similar to the technical committees, the Cornerstones committees must secure their own funds, operate within their own budget, and deliver a budget to the Treasurer within the first week of the academic year.
    \item Cornerstones committees may request financial assistance from the organization as a whole due to their inclusion of all members.
\end{itemize}


\article{Technical Committees}
\label{art:technical}

\section{Basic Powers}
\label{sec:tech_pow}
\begin{itemize}
    \item Standing technical committees may be established by an elected officer majority vote (EOMV) to be founded under the goals and leadership of the committee chair that the elected officers choose.
    \item These standing committees may be disbanded upon mutual agreement of the President and the corresponding committee chair so long as the latter position is filled or upon an Executive Committee supermajority vote (ECSV).
    \item The practice of establishing co-chairs of technical committees is discouraged.
    \item Should multiple people chair a technical committee, the single vote of the committee will be decided by a runoff between all the co-chairs prior to tallying the Executive Committee votes.
    \item Vice chairs are allowed but will not be given any voting rights at the Executive Committee level.
\end{itemize}

\section{Internal Governance}
\label{sec:tech_govern}
\begin{itemize}
    \item Technical committees must select their own chair by election time each year.
	\begin{itemize}
	\item Under extenuating circumstances, a technical committee may request the committee chair be removed after elections, but before the start of the academic year by an ECSV. 
	\item The technical committee may request an extension to selecting their own chair subject to approval by an EOMV.
	\end{itemize}
    \item The committee chair shall establish a method of governance for the committee consistent with this Constitution.
    \item Each committee is responsible for outlining its own goals and purposes and drawing membership from the Student Branch as a whole.
    \item Formal procedures must be added to bylaws specific to the technical committee as mentioned in Article \ref{art:amend}, Section \ref{sec:amend_techbylaw}.
\end{itemize}

\section{Internal Operations}
\label{sec:tech_operate}
\begin{itemize}
    \item Notwithstanding explicit constitutional provisions, technical committees shall have the ability to operate independently of approval from the Executive Committee but with operations visible to the elected officers.
    \item Each technical committee may determine its own requirements for committee membership and schedule for events and meetings.
    \item Each committee chair shall report the internal leadership, activities, meetings, and communications of its committee thoroughly and promptly to the officers and other committee chairs.
    \item Furthermore, the committee chair shall make an effort to educate members about the opportunities the Cornerstones committees and international IEEE offer.
\end{itemize}

\section{Technical Committee Finances}
\label{sec:tech_budget}
\begin{itemize}
    \item Each committee is expected to operate within its own budget as part of the operations of IEEE.
    \item Each committee should advance its budget to the IEEE Treasurer within the first week of each new academic year.
    \item In accordance with the guidelines imposed by the Business Office of Student Organizations, the Treasurer must make deposits and authorize expenditures for all committees.
    \item Technical committees automatically forfeit a funding claim through the Rakos Funding Distribution Act if they fail to report it within 21 days of the Treasurer informing the technical committee of unlabeled funding. This forfeiture can not occur within the first 120 days of the posting date of the income line item in the account. Any singular forfeiture can be overturned by an ECSV.
    \item Technical Committees are encouraged to report any expected income in advance.
\end{itemize}

\section{Limits to Independent Representation}
\label{sec:tech_limrepres}
\begin{itemize}
    \item Technical committees may participate in competitions, exhibitions, and the like under their own names rather than that of the Purdue IEEE Student Branch if the organization as a whole is not actively participating in such events.
    \item The committee chair is responsible for securing permission from the President and Primary Advisor if needed.
    \item The Executive Committee should restrain itself from disrupting prior arrangements of technical committees.
    \begin{itemize}
        \item If the Executive Committee or a group of technical committees is considering participation in a competition, exhibition, or the like for which a single technical committee is considering, the Executive Committee has seven days from the time of initial notification to decide via an ECSV on the participation of Purdue IEEE Student Branch or committees as a whole.
        \item Should such a vote pass, then the responsibility for communication, planning, execution, and expenses regarding the event shifts over to the Executive Committee.
        \item In all other circumstances, the right of the technical committee to independent representation is upheld.
    \end{itemize}
    \item No technical committee chair may include the word ``President'' in his or her internal title but is otherwise free to represent himself or herself when the technical committee operates under its own name.
    \item No technical committee is allowed to use ``callout'' to describe its introductory meeting for new members and required to instead use the phrase ``information session.''
\end{itemize}

\section{Joint Projects}
\label{sec:tech_joint}
\begin{itemize}
    \item Joint projects between student organizations, academic bodies, or off-campus entities of a scope in time, labor, or cost that the Executive Committee deems significant shall be formalized in a project charter.
    \item Project charters should cover the allotment of participating members, finances, tools, workspace, and other resources.
    \item Concerns such as project ownership and duration should be addressed within.
    \item The Secretary shall make the adopted project charter available for the Executive Committee to view for the duration it is in effect.
    \item Project charters should be signed by the appropriate representatives of the Executive Committee following passage and the corresponding governing bodies of other student organizations or entities.
\end{itemize}

\section{Technical Committee Good Standing}
\label{sec:tech_goodstand}
\begin{itemize}
    \item A technical committee may lose its good standing with the Student Branch by an EOMV.
    \item Reasons for losing good standing include but are not limited to: continued member absence from IEEE meetings, closing operations from the view of elected officers, willful disregard of the Constitution, continued bankruptcy of its accumulated funds, aggressive actions against other committees, or failure to follow established technical committee bylaws.
    \item A technical committee that has lost its good standing will forfeit representation on the Executive Committee, be denied making future purchases except those individually approved by the Treasurer, be prevented from bidding for unallocated funding or resources in possession of the organization as a whole, and lose access to room reservations and other systems used on behalf of IEEE.
    \item The Vice President of Technical Committees must approve of technical committee restructuring actions to correct the reasons that led to loss of good standing.
    \item A technical committee will regain its good standing after another EOMV.
\end{itemize}

\section{Utilization of the Purdue IEEE Workspace}
\label{sec:tech_workspace}
\begin{itemize}
    \item This section is in effect provided that Purdue IEEE Student Branch hold lease of a communal workspace in EE 020.
    \item If and only if a committee is in good standing with the Student Branch, it is permitted to reserve EE 020.
    \item Individuals are permitted to use communal equipment in EE 020 if and only if a.) are in good standing with Purdue IEEE Student Branch and and (b.) have completed all safety certifications specified either by School of Electrical and Computer Engineering management or by the Vice President of Technical Committees.
    \item Workspace users may request that the Vice President of Technical Committees (or somebody appointed by the Vice President of Technical Committees) remove nonfunctional equipment from EE 020. The Vice President of Technical Committees may return equipment if he or she determines that key functionalities have been restored.
    \item Committees that utilize EE 020 for long-term storage have the following responsibilities:
    \begin{enumerate}
        \item Submitting an inventory in the specified format to the Purdue IEEE Executive Committee by the third week of the academic year.
        \item Labelling storage containers and items valued over USD 100 not intended for communal use.
        \item Placing a safety data sheet (SDS) sticker completed with the appropriate NFPA 704 classifications on the containers of any chemical agent that satisfies any of the following: \emph{Instability/Reactivity $>$ 1, Health $>$ 1, Flammability $>$ 1}, has a special notice (eg. COR, OX, ALK).
        \item Items with \emph{Flammability $>$ 2} must be stored in the fire safety cabinet inside the workspace when not in use, with the door of the cabinet fully shut.
        \begin{itemize}
            \item \emph{Exception:} Items with the special notice OX or \st{W} or \emph{Instability/Reactivity $>$ 1} \textbf{may not} be stored in the fire safety cabinet and require special storage instructions from the Vice President of Technical Committees. 
        \end{itemize}
    \end{enumerate}
    \item Individuals are banned from using any chemical substance with a vapor pressure exceeding 100 kPa at 25$^{\circ}$C inside EE 020 (e.g., spray paint).
    \item Committees that perform work at any time in EE 020 must:
    \begin{enumerate}
        \item Reserve the room for the committee in the manner dictated by the Secretary whenever five or more people are working on projects for a given committee.
        \item Clear off the designated temporary work tables before leaving the workspace.
    \end{enumerate}
    \item Individuals operating rotary power tools, sanders, drills, or saws in EE 020 at any time must also do the following:
    \item
    \begin{enumerate}
        \item Perform all work with these tools while wearing ANSI-approved safety glasses or safety goggles.
        \item Complete any posted training for these tools prior to starting work.
        \item Report any tool needing maintenance and any maintenance performed to the Vice President of Technical Committees.
    \end{enumerate}
    \item Furthermore, individuals are strongly encouraged to seek help while working with unfamiliar equipment and not to work alone in EE 020.
    \item Individuals must follow proper protocol while unlocking and locking the door to EE 020. 
    \item Damages to communal property knowingly caused by members of a given committee may be billed to that committee with Treasurer approval.
    \item If deemed necessary, Executive Committee action can be taken against individuals or committees that are deemed to be in infraction with the protocols stated in this section with a mandate that corrective action be taken by a specified date. Possible actions are (in increasing level of severity) as follows:
    \begin{enumerate}
        \item Emailed warning with the Executive Committee carbon copied; can be sanctioned by any elected officer.
        \item Temporary access restriction; can be sanctioned by the Vice President of Technical Committees.
        \item Loss of good standing; can be sanctioned through an EOMV.
        \item Ban from using EE 020 for the rest of the academic year and removal of items from EE 020; can be sanctioned through an ECMV.
    \end{enumerate}
\end{itemize}

\article{Event Committees}
\label{art:event}

\section{Basic Powers}
\label{sec:event_pow}
\begin{itemize}
    \item Standing event committees may be established by an Executive Committee supermajority vote (ECSV) to be founded under the goals and leadership of the committee chair that the Executive Committee chooses.
    \item These standing committees may be disbanded upon an Executive Committee supermajority vote (ECSV) with assent from all external entities jointly hosting the event committee.
    \item The practice of establishing co-chairs of event committees is disallowed.
    \item Vice chairs are allowed but will not be given any voting rights at the Executive Committee level.
\end{itemize}

\section{Internal Governance}
\label{sec:event_govern}
\begin{itemize}
    \item Event committees must select their own chair by election time each year.
	\begin{itemize}
	\item Under extenuating circumstances, an event committee may request the committee chair be removed after elections, but before the start of the academic year by an ECSV. 
	\item The event committee may request an extension to selecting their own chair subject to approval by an EOMV.
	\end{itemize}
    \item The committee chair shall establish a method of governance for the committee consistent with this Constitution.
    \item Each committee is responsible for outlining its own goals and purposes and drawing membership from the Student Branch as a whole.
    \item Event committees may distinguish between volunteer leaders for their continuous event who must be members of the Student Branch as a whole, employed leaders who are only encouraged to be members of the Student Branch, and event attendees with no restrictions on membership status.
    \item Formal procedures must be added to bylaws specific to the event committee as mentioned in Article \ref{art:amend}, Section \ref{sec:amend_techbylaw}.
\end{itemize}

\section{Internal Operations}
\label{sec:event_operate}
\begin{itemize}
    \item Event committees operate under the jurisdiction of the President with operations fully visible to the Executive Committee and any external entity with joint responsibility for hosting the event committee.
    \item Each event committee may determine its own requirements for committee membership and schedule for events and meetings.
    \item Each committee chair shall report the internal leadership, activities, meetings, and communications of its committee thoroughly and promptly to the officers and other committee chairs.
\end{itemize}

\section{Event Committee Budgets}
\label{sec:event_budget}
\begin{itemize}
    \item Each committee is expected to operate within its own budget as part of the operations of IEEE.
    \item Each committee should advance its budget to the IEEE Treasurer within the first week of each new academic year.
    \item In accordance with the guidelines imposed by the Business Office of Student Organizations, the Treasurer must make deposits and authorize expenditures for all committees using organization funding.
    \item Alternatively, an external entity with joint responsibility for hosting the event committee may impose its own regulations for budgets without the input of the Treasurer should that external entity provide for the entirety of the event committee funding alone.
\end{itemize}

\section{Necessary Inclusion of IEEE}
\label{sec:inclusion}
\begin{itemize}
    \item Event committees must operate under the name of Purdue IEEE Student Branch for all events.
    \item No event committee chair may include the word ``President'' in his or her internal title.
    \item Joint hosting of an event committee with external entities should be formalized in a project charter as detailed in \ref{art:technical}, Section \ref{sec:tech_joint} or codified in the event committee bylaws given the continuous nature of event committee functions.
    \item The Secretary shall make the adopted project charter available for the Executive Committee to view for the duration it is in effect.
    \item No event committee is allowed to use ``callout'' to describe its introductory meeting for new members and required to instead use the phrase ``information session.''
\end{itemize}

\section{Event Committee Good Standing}
\label{sec:event_goodstand}
\begin{itemize}
    \item An event committee may lose its good standing with the Student Branch by an EOMV.
    \item Reasons for losing good standing include but are not limited to: closing operations from the view of elected officers, willful disregard of the Constitution, continued bankruptcy of event committee funds, aggressive actions against other committees, mishandling of employed leaders, or failure to follow established event committee bylaws.
    \item An event committee that has lost its good standing will forfeit representation on the Executive Committee, be denied making future purchases from IEEE funds, be prevented from bidding for unallocated funding or resources in possession of the organization as a whole, and lose access to room reservations and other systems used on behalf of IEEE.
    \item The President must approve of event committee restructuring actions to correct the reasons that led to loss of good standing.
    \item An event committee will regain its good standing after another EOMV or upon remediation with the President and representatives from all external entities jointly hosting the event committee.
\end{itemize}


\article{Meetings}
\label{art:meet}

The Student Branch shall hold regular and special meetings beyond the meetings of the technical committees at such places and time as designated by the President, with a minimum of five meetings per academic year.


\article{Student Branch Chapters}
\label{art:sbc}

\textit{Student Branch Chapters of international IEEE Societies shall be organized as technical committees under the same provisions as in Article \ref{art:technical} without being regarded as inherently superior or inferior to other technical committees. Given the international prestige of Student Branch Chapters, these special items shall also apply:}

\section{Chapter Faculty Advisor}
\label{sec:sbc_cfa}
\begin{itemize}
    \item Student Branch Chapters shall select a Chapter Faculty Advisor from the faculty of Purdue University who is a member of both IEEE and the corresponding society.
    \item This individual may also act as the Primary Advisor or a Secondary Advisor of the student organization or the Branch Counselor.
\end{itemize}

\section{Student Branch Chapter Chair}
\label{sec:sbc_chair}
\begin{itemize}
    \item The members of each Student Branch Chapter shall elect a chair for the technical committee named after the respective IEEE Society.
    \item This person will be titled with the name or abbreviation of their technical committee followed by the word ``Chair'' and must be decided prior to the elections of the Student Branch as a whole.
    \item The candidate eligibility, nominations, member voting, and tie-breaking shall follow the requirements of Article \ref{art:elect} of this Constitution.
    \item All voting of the Student Branch Chapters shall take effect on or before \datetermstart{} as determined by the committee.
    \item Each Student Branch Chapter Chair must enroll as an international IEEE member and as a member of their respective international IEEE Society for the duration of their term in office to be in good standing with the student branch.
\end{itemize}

\section{Student Branch Chapter Operations}
\label{sec:sbc_operate}
\begin{itemize}
    \item Each society chair shall have jurisdiction over the affairs of his or her Student Branch Chapter and preside over its meetings.
    \item The society chair may assume all other necessary executive duties or assign them to other society executives that the society chair appoints or brings to election.
    \item The society chair is responsible for any reporting requirements specific to the corresponding IEEE Society and for consulting with the Chapter Faculty Advisor.
    \item Subcommittees may also be formed under directors to accomplish these purposes.
\end{itemize}

\section{Student Branch Chapter Bylaws}
\label{sec:sbc_bylaw}
\begin{itemize}
    \item All Student Branch Chapters must operate in accordance with the Constitution and Bylaws of their respective international IEEE Society and the rest of this Constitution.
    \item Each IEEE Society should follow the purposes laid out in Article II, with additional purposes, local society dues provisions, officer structure, and requirements put into technical committee bylaws specific to that society as required in Article \ref{art:amend}, Section \ref{sec:amend_techbylaw}.
\end{itemize}

\section{Connection to Rest of IEEE}
\label{sec:sbc_connect}
\begin{itemize}
    \item Student Branch Chapters have the independence and expectations of operations stated in Article \ref{art:technical}.
    \item New Student Branch Chapters are encouraged to form under Purdue IEEE Student Branch rather than pursue independent student organization status.
\end{itemize}

\section{Special Rules for Dissolution}
\label{sec:sbc_csociety}
\begin{itemize}
    \item A Student Branch Chapter of an IEEE Society shall not be disbanded without approval of the Chair of that Society.
    \item The above clause does not automatically apply to any other non-society committees belonging to Purdue IEEE Student Branch.
\end{itemize}


\article*{Signatures}

\vspace{0.5in}
\begin{tabular}{ll}
    \makebox[3.0in]{\hrulefill} & \makebox[1.5in]{\hrulefill} \\
    Student Activities and Organizations & Date of Recognition \vspace{1.0in} \\
    \makebox[3.0in]{\hrulefill} & \makebox[1.5in]{\hrulefill} \\
    Primary Advisor -- Christopher Brinton, Ph.D. & Date of Adoption \vspace{1.0in} \\
    \makebox[3.0in]{\hrulefill} & {} \\
    President of IEEE -- Swaggat Bhattacharyya & {}
\end{tabular}

\setcounter{tocdepth}{1}
\listoftodos % Comment out when not editing


\end{document}
