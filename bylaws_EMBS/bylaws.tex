% Preamble
\documentclass[12pt]{constitution}
\usepackage{mathpazo, graphicx}
\graphicspath{{figures/}}
\usepackage[draft]{todonotes} % Show comments in PDF
%\usepackage[disable]{todonotes} % Hide comments from PDF

% Metadata
\title{Engineering in Medicine and Biology Society (EMBS) Chapter Bylaws}
\author{Technical Committee of Purdue IEEE Student Branch}
\date{}
\newcommand{\datetermstart}{June 1\textsuperscript{st}} % Place term start date here
\newcommand{\dateannualreportsubmit}{May 1\textsuperscript{st}} % Place IEEE Annual Student Report submission deadline here
\newcommand{\datefiscalstart}{August 1\textsuperscript{st}} % Place fiscal year start date here
\newcommand{\dateelectionsheld}{May 1\textsuperscript{st}} % Place last day elections may be held here


\begin{document}
% Title Option #1: Title Page and Table of Contents Separate
%\maketitle
%\setcounter{tocdepth}{0} % Change to 0 for articles only or 1 for articles and sections
%\tableofcontents
%\newpage

% Title Option #2: Title at Start of Table of Contents
\titlecontentspage
\newpage


\article{Name and Purpose}
\label{art:namepurp}

Purdue IEEE Engineering in Medicine and Biology Society (EMBS) exists as a Student Branch Chapter under the Purdue IEEE Student Branch. In the spirit of the purpose of the organization, the mission of IEEE EMBS is to advance the field of engineering in healthcare by holding events to inform members of applications of biomedical technology and by providing experience and exposure to technology used in both engineering and healthcare. All EMBS members are required to follow the rules established in these bylaws in addition to the Constitution of IEEE.

\article{Membership}
\label{art:member}

The members of Purdue IEEE EMBS are those from the Purdue IEEE Student Branch who voluntarily associate themselves with this committee. Active members will be those considered to attend at least three events specific to EMBS each academic term in addition to prompt payment of all local IEEE dues and other requirements for good standing in Purdue IEEE Student Branch. The undergraduate and graduate student members that are active members of Purdue IEEE EMBS are entitled to the full rights and voting privileges of members. Membership in IEEE EMBS at the international level is separate from local membership and alone is not sufficient to be considered active in the local chapter.

The Chapter Faculty Advisor will be a member of Purdue IEEE EMBS that is part of the faculty of Purdue University active in both IEEE and IEEE EMBS in particular. The leadership of Purdue IEEE EMBS has the right to extend honorary membership to any person. The Chapter Faculty Advisor and all honorary members lack the ability to vote.

\article{Leadership}
\label{art:leader}

The Purdue IEEE EMBS leadership shall consist of the following voting individuals:
\begin{itemize}
    \item EMBS Chair
    \item EMBS Vice Chair
    \item EMBS Electrical Lead
    \item EMBS Mechanical Lead
    \item EMBS Programming Lead
    \item Approved subcommittee directors (e.g., project leadership)
\end{itemize}

The EMBS Chair alone will exercise the ability to vote within the Executive Committee of Purdue IEEE Student Branch.

The EMBS Chair will preside over all meetings of Purdue IEEE EMBS and transact all business necessary to address the administrative needs of the society chapter. He or she shall be the point of contact with IEEE EMBS, as well as Purdue IEEE regarding operations within the society chapter. He or she shall strive to meet the needs of other EMBS members and has the ability to delegate tasks as necessary to achieve the purposes of the society chapter.

The EMBS Vice Chair serves as the junior executive leader of Purdue IEEE EMBS and shall set forth a technical direction for the academic year, oversee the success of any technical projects, and otherwise assist the EMBS Chair with his or her duties. The EMBS Chair is responsible for informing the society of student branch activities.  The EMBS Vice Chair shall be the point of contact between the society chapter and the faculty at Purdue University, and will be tasked with meeting with faculty members for planning, marketing, and executing social activities that provide exposure for members in the specialized fields of biological or biomedical engineering along with the EMBS Chair.

The EMBS Electrical, Mechanical, and Programming Leads are responsible for conducting and providing subsequent goals for members to follow throughout the course of meetings. Moreover, they are held to the responsibility of guiding and teaching members that may have questions or doubts concurrent to the respective project. These leads must communicate to one another and the EMBS Chair and Vice Chair about each lesson plan conducted during the meeting. They must act in accordance with members’ interests and further work together in best interest towards the final project.

\article{Procedure for Decisions}
\label{art:decide}

The ideas and opinions of the collective members of Purdue IEEE EMBS should guide all decisions within the committee. As the leader who balances the needs of the members, the EMBS Chair shall have the authority to make decisions on behalf of the entire society chapter. However, any other leader in Purdue IEEE EMBS may call for a vote among the leadership on the issue in question. A simple majority of Purdue IEEE EMBS leaders voting in the manner decided by the EMBS Chair will determine what the decision shall be if it differs from the original decision from the EMBS Chair. There is no minimum on leaders present for a quorum. Proxy voting is disallowed.

The EMBS Chair shall have the power to decide access to committee resources in a manner consistent with these bylaws. Workspace granted specifically to Purdue IEEE EMBS by faculty or staff shall be in the stewardship of the EMBS Chair and Chapter Faculty Advisor. The EMBS Chair will have full control over the budgeting of all expenses and approval of all purchases, while the EMBS Vice Chair has provisional control over the approval of all purchases. These two privileges may be extended to others by order of the EMBS Chair. The EMBS Chair and Vice -Chair will create the EMBS budget for the fiscal year by the first day of the academic year such that it can be delivered to the IEEE Treasurer within the span of one week in accordance with the first week of the academic year in accordance with Article X, Section 4 of the Constitution of IEEE. The IEEE Treasurer will only reimburse purchases made with approval from Purdue IEEE EMBS.

\article{Elections to and Departures from Leadership}
\label{art:electdepart}

The EMBS Chair and EMBS Vice Chair shall be elected among the voting Purdue IEEE EMBS members. The EMBS Electrical, EMBS Mechanical, and EMBS Programming leads shall be appointed by the newly elected EMBS Chair and EMBS Vice Chair. Should the EMBS Chair and Vice-Chair disagree on any of the appointed positions, the previous year’s Chair and Vice-Chair shall advise and give input. The EMBS Chair and other elected leaders should enroll as members of IEEE and IEEE Engineering in Medicine and Biology Society additionally for the duration of service. Elections should occur at least once per academic year at least one day prior to the elections of the Purdue IEEE Student Branch. The term of office is coincident to the start of the term of office of the Purdue IEEE Student Branch elected officers.

The EMBS Chair shall decide a teller for the elections. The teller shall have the authority over vote counting and the ability to decide which votes on a ballot are valid. The specifics of nominations, voting process, and tie breaking shall be given in the Constitution of IEEE.

A leader of Purdue IEEE EMBS may resign their role at any time. Two-thirds of active EMBS members may sign a petition requesting the initiation of the recall process for any specific leader. Active members will be those considered to attend at least three events specific to EMBS each academic term in addition to prompt payment of all local IEEE dues and other requirements for good standing in Purdue IEEE Student Branch. The recall is successful when a simple majority of the Purdue IEEE EMBS leaders vote in favor of recalling the leader. Vacancies in office shall be handled in the manner described below.

The EMBS leadership shall normally be elected by the method outlined in the Constitution of IEEE. The EMBS Chair and EMBS Vice Chair positions should be occupied at all times by different elected individuals. In the event of a vacancy of the EMBS Vice Chair prior to elections, the EMBS Chair shall choose a successor who holds the office until an election may be held. In the vent of a vacancy of the EMBS Chair prior to elections, the EMBS Vice Chair shall immediately assume the role of EMBS Chair and plan for a speedy election of the leadership within 30 days. Should oth the EMBS Chair and EMBS Vice Chair simultaneously be unoccupied, then the Purdue IEEE Vice President shall work with the remaining leadership to elect those roles to ensure the livelihood of the committee.

\article{Amendments to Bylaws}
\label{art:amend}

Any amendments to these bylaws shall be proposed by the EMBS Chair. The bylaws shall pass with a two-thirds vote of the leadership. Passed bylaws shall take effect when recognized by the Executive Committee of Purdue IEEE Student Branch as described in the Constitution of IEEE.

\article{Society Dues}
\label{art:dues}

Purdue IEEE EMBS chooses to forego dues specific to this local society chapter beyond those required by the student branch as a whole. A reversal of this position requires an amendment to these bylaws.

\article{Subcommittees}
\label{art:subcommittee}

The EMBS Chair has the right to form subcommittees under a named director in Purdue IEEE EMBS to accomplish mutually agreed upon goals. The continued existence of the subcommittee and the position of the director are at the discretion of the EMBS Chair. All operations of the subcommittee are under the purview of the EMBS Chair and EMBS Vice Chair. A majority vote of the existing EMBS leadership is needed to add any subcommittee director to EMBS leadership.

\vspace{12pt}
\hrule

\textit{Effective: September 30\textsuperscript{th}, 2022}


\setcounter{tocdepth}{1}
%\listoftodos % Comment out when not editing


\end{document}
