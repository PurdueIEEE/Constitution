% Preamble
\documentclass[12pt]{constitution}
\usepackage{mathpazo, graphicx}
\graphicspath{{figures/}}
\usepackage[draft]{todonotes} % Show comments in PDF
%\usepackage[disable]{todonotes} % Hide comments from PDF

% Metadata
\title{Engineering in Medicine and Biology Society (EMBS) Chapter Bylaws}
\author{Technical Committee of Purdue IEEE Student Branch}
\date{}
\newcommand{\datetermstart}{June 1\textsuperscript{st}} % Place term start date here
\newcommand{\dateannualreportsubmit}{May 1\textsuperscript{st}} % Place IEEE Annual Student Report submission deadline here
\newcommand{\datefiscalstart}{August 1\textsuperscript{st}} % Place fiscal year start date here
\newcommand{\dateelectionsheld}{May 1\textsuperscript{st}} % Place last day elections may be held here


\begin{document}
% Title Option #1: Title Page and Table of Contents Separate
%\maketitle
%\setcounter{tocdepth}{0} % Change to 0 for articles only or 1 for articles and sections
%\tableofcontents
%\newpage

% Title Option #2: Title at Start of Table of Contents
\titlecontentspage
\newpage


\article{Name and Purpose}
\label{art:namepurp}

Purdue IEEE Engineering in Medicine and Biology Society (EMBS) exists as a Student Branch Chapter under the Purdue IEEE Student Branch. In the spirit of the purpose of the organization, the mission of IEEE EMBS is to advance the field of engineering in the healthcare field by holding events to inform members of applications of biomedical technology and by providing experience and exposure to technology used in both engineering and healthcare. All EMBS members are required to follow the rules established in these bylaws in addition to the Constitution of IEEE.


\article{Membership}
\label{art:member}

The members of Purdue IEEE EMBS are those from the Purdue IEEE Student Branch who voluntarily associate themselves with this committee. Active members will be those considered to attend at least three events specific to EMBS each academic term in addition to prompt payment of all local IEEE dues and other requirements for good standing in Purdue IEEE Student Branch. The undergraduate and graduate student members that are active members of Purdue IEEE EMBS are entitled to the full rights and voting privileges of members. Membership in IEEE EMBS at the international level is separate from local membership and alone is not sufficient to be considered active in the local chapter.

The Chapter Faculty Advisor will be a member of Purdue IEEE EMBS that is part of the faculty of Purdue University active in both IEEE and IEEE EMBS in particular. The leadership of Purdue IEEE EMBS has the right to extend honorary membership to any person. The Chapter Faculty Advisor and all honorary members lack the ability to vote.


\article{Leadership}
\label{art:leader}

\todo[inline,backgroundcolor=red!25]{MRH: So I heard that new leadership positions have been added and that many of the existing positions are being redefined.}
The Purdue IEEE EMBS leadership shall consist of the following voting individuals:
\begin{itemize}
    \item EMBS Chair
    \item EMBS Vice Chair
    \item EMBS Student Liaison
    \item EMBS Industrial Liaison
    \item EMBS Faculty Liaison
    \item Approved subcommittee directors (e.g., project leadership)
\end{itemize}

The EMBS Chair alone will exercise the ability to vote within the Executive Committee of Purdue IEEE Student Branch. The EMBS Industrial Liaison, EMBS Student Liaison, and EMBS Faculty Liaison shall be the \textit{ex officio} Professional Committee representative, Social Committee representative, and Learning Committee representative respectively to the Cornerstones committees of Purdue IEEE Student Branch on behalf of EMBS members. \todo[backgroundcolor=red!25]{MRH: The correct correspondence of liaisons to Cornerstones committee representatives from EMBS should be checked.}

The EMBS Chair will preside over all meetings of Purdue IEEE EMBS and transact all business necessary to address the administrative needs of the society chapter. He or she shall strive to meet the needs of other EMBS members and has the ability to delegate tasks as necessary to achieve the purposes of the society chapter.

The EMBS Vice Chair serves as the junior executive leader of Purdue IEEE EMBS and shall set forth a technical direction for the academic year, ensure that events are planned for successful outcomes, be responsible for all email correspondence (sending and responding to emails), maintain the mailing list, maintain the Slack channels, and reserve rooms for meetings and events, and otherwise assist the EMBS Chair with his or her duties.

The EMBS Industrial Liaison shall be an industrial point of contact for the society chapter and will work with companies, faculty, and campus entities to acquire funding for events and hold seminars of interest to EMBS members.

The EMBS Student Liaison will create plans for extended learning opportunities for EMBS members that explore topics in depth or impart valuable skills to those in attendance.

The EMBS Faculty Liaison shall be a point of contact between the society chapter and the faculty at Purdue University, and will be tasked with meeting with faculty members for planning, marketing, and executing social activities that provide exposure for members in the specialized fields of biological or biomedical engineering.

The Chapter Faculty Advisor for Purdue IEEE EMBS shall act as a non-voting member of the leadership. He or she shall guide the leadership to best serve the needs of the healthcare and medicine community on campus and throughout IEEE, aid in the continuity of activities, and connect members to opportunities in academia or industry. The EMBS Chair is responsible for informing the Chapter Faculty Advisor of society chapter activities.


\article{Procedure for Decisions}
\label{art:decide}

The ideas and opinions of the collective members of Purdue IEEE EMBS should guide all decisions within the committee. As the leader who balances the needs of the members, the EMBS Chair shall have the authority to make decisions on behalf of the entire society chapter. However, any other leader in Purdue IEEE EMBS may call for a vote among the leadership on the issue in question. A simple majority of Purdue IEEE EMBS leaders voting in the manner decided by the EMBS Chair will determine what the decision shall be if it differs from the original decision from the EMBS Chair. There is no minimum on leaders present for a quorum. Proxy voting is disallowed.

The EMBS Chair shall have the power to decide access to committee resources in a manner consistent with these bylaws. Workspace granted specifically to Purdue IEEE EMBS by faculty or staff shall be in the stewardship of the EMBS Chair and Chapter Faculty Advisor. The EMBS Vice Chair will have full control over the budgeting of all expenses and approval of all purchases. These two privileges may be extended to others by order of the EMBS Chair and Vice Chair, respectively\todo[backgroundcolor=red!25]{MRH: The word ``respectively'' is a bit unclear in meaning for this context.}. The IEEE Treasurer will only reimburse purchases made with approval from Purdue IEEE EMBS.
\todo[inline,backgroundcolor=red!25]{MRH: How is the EMBS Vice Chair going to ensure the EMBS budget for the fiscal year is delivered to the IEEE Treasurer within the first week of the academic year in accordance with Article X, Section 4 of the Constitution of IEEE?}


\article{Elections to and Departures from Leadership}
\label{art:electdepart}

The EMBS Chair, EMBS Vice Chair, EMBS Industrial Liaison, EMBS Student Liaison, and EMBS Faculty Liaison shall be elected among the voting Purdue IEEE EMBS members. The EMBS Chair and other elected leaders should enroll as members of IEEE and IEEE Engineering in Medicine and Biology Society additionally for the duration of service. Elections should occur at least once per academic year at least one day prior to the elections of the Purdue IEEE Student Branch. The term of office is coincident to the start of the term of office of the Purdue IEEE Student Branch elected officers.

The EMBS Chair shall decide a teller for the elections. The teller shall have the authority over vote counting and the ability to decide which votes on a ballot are valid. The specifics of nominations, voting process, and tie breaking shall be given in the Constitution of IEEE.

A leader of Purdue IEEE EMBS may resign their role at any time. Two-thirds of EMBS members \todo[backgroundcolor=red!25]{MRH: In practice, how are you going to determine what two-thirds of the total EMBS membership is?}may sign a petition requesting the initiation of the recall process for any specific leader. The recall is successful when a simple majority of the Purdue IEEE EMBS leaders vote in favor of recalling the leader. Vacancies in office shall be handled in the manner described below.

The EMBS leadership shall normally be elected by the method outlined in the Constitution of IEEE. The EMBS Chair and EMBS Vice Chair positions should be occupied at all times by different elected individuals. In the event of a vacancy of the EMBS Vice Chair prior to elections, the EMBS Chair shall choose a successor who holds the office until an election may be held. In the event of a vacancy of the EMBS Chair prior to elections, the EMBS Vice Chair shall immediately assume the role of EMBS Chair and plan for a speedy election of the leadership within 30 days. Should both the EMBS Chair and EMBS Vice Chair simultaneously be unoccupied, then the Purdue IEEE Vice President shall work with the remaining leadership to elect those roles to ensure the livelihood of the committee. Extended vacancies of the EMBS Industrial Liaison, EMBS Student Liaison, and EMBS Faculty Liaison are allowed, provided that the remaining leadership performs the duties of those positions.


\article{Amendments to Bylaws}
\label{art:amend}

Any amendments to these bylaws shall be proposed by the EMBS Chair. The bylaws shall pass with a two-thirds vote of the leadership. Passed bylaws shall take effect when recognized by the Executive Committee of Purdue IEEE Student Branch as described in the Constitution of IEEE.


\article{Society Dues}
\label{art:dues}

Purdue IEEE EMBS chooses to forego dues specific to this local society chapter beyond those required by the student branch as a whole. A reversal of this position requires an amendment to these bylaws.


\article{Subcommittees}
\label{art:subcommittee}

The EMBS Chair has the right to form subcommittees under a named director in Purdue IEEE EMBS to accomplish mutually agreed upon goals. The continued existence of the subcommittee and the position of the director are at the discretion of the EMBS Chair. All operations of the subcommittee are under the purview of the EMBS Chair and EMBS Vice Chair. A majority vote of the existing EMBS leadership is needed to add any subcommittee director to EMBS leadership.


\todo[inline,backgroundcolor=red!25]{MRH: How is EMBS going to ensure that there are enough representatives sent to the Cornerstones committees if additional restrictions are imposed in the future?}

\vspace{12pt}
\hrule

\textit{Effective: February 24\textsuperscript{th}, 2017}


\setcounter{tocdepth}{1}
\listoftodos % Comment out when not editing


\end{document}