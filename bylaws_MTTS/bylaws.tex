% Preamble
\documentclass[12pt]{constitution}
\usepackage{mathpazo, graphicx}
\graphicspath{{figures/}}
\usepackage[draft]{todonotes} % Show comments in PDF
%\usepackage[disable]{todonotes} % Hide comments from PDF

% Metadata
\title{Microwave Theory and Techniques Society (MTT-S) Chapter Bylaws}
\author{Technical Committee of Purdue IEEE Student Branch}
\date{}
\newcommand{\datetermstart}{June 1\textsuperscript{st}} % Place term start date here
\newcommand{\dateannualreportsubmit}{May 1\textsuperscript{st}} % Place IEEE Annual Student Report submission deadline here
\newcommand{\datefiscalstart}{August 1\textsuperscript{st}} % Place fiscal year start date here
\newcommand{\dateelectionsheld}{May 1\textsuperscript{st}} % Place last day elections may be held here


\begin{document}
% Title Option #1: Title Page and Table of Contents Separate
%\maketitle
%\setcounter{tocdepth}{0} % Change to 0 for articles only or 1 for articles and sections
%\tableofcontents
%\newpage

% Title Option #2: Title at Start of Table of Contents
\titlecontentspage
\newpage


\article{Name and Purpose}
\label{art:namepurp}

Purdue IEEE Microwave Theory and Techniques Society (MTT-S) exists as a Student Branch Chapter under the Purdue IEEE Student Branch. In the spirit of the purpose of the organization, the mission of IEEE MTT-S is to advance the field of microwave engineering by holding events to inform members about the theory and applications of high-frequency technology. All MTT-S members are required to follow the rules established in these bylaws in addition to the Constitution of IEEE.


\article{Membership}
\label{art:member}

The members of Purdue IEEE MTT-S are those from the Purdue IEEE Student Branch who voluntarily associate themselves with this committee. Active members are those who attend at least two events specific to MTT-S each academic term in addition to prompt payment of all local IEEE dues and other requirements for good standing in Purdue IEEE Student Branch. The undergraduate and graduate student members that are active members of Purdue IEEE MTT-S are entitled to the full rights and voting privileges of members. Membership in IEEE MTT-S at the international level is separate from local membership and alone is not sufficient to be considered active in the local chapter.

The Chapter Faculty Advisor will be a member of Purdue IEEE MTT-S that is part of the faculty of Purdue University active in both IEEE and IEEE MTT-S in particular. The leadership of Purdue IEEE MTT-S has the right to extend honorary membership to any person. The Chapter Faculty Advisor and all honorary members lack the ability to vote.


\article{Leadership}
\label{art:leader}

The Purdue IEEE MTT-S leadership shall consist of the following voting individuals:
\begin{itemize}
    \item MTT-S Chair
    \item MTT-S Vice Chair
    \item MTT-S Sponsorship Coordinator
    \item MTT-S Workshop Coordinator
    \item MTT-S Event Coordinator
    \item Approved subcommittee directors
\end{itemize}

The MTT-S Chair alone will exercise the ability to vote within the Executive Committee of Purdue IEEE Student Branch. The MTT-S Sponsorship Coordinator, MTT-S Workshop Coordinator, and MTT-S Event Coordinator shall be the \textit{ex officio} Professional Committee representative, Learning Committee representative, and Social Committee representative respectively to the Cornerstones committees of Purdue IEEE Student Branch on behalf of MTT-S members.

The MTT-S Chair will preside over all meetings of Purdue IEEE MTT-S and transact all business necessary to address the administrative needs of the society chapter. He or she shall strive to meet the needs of other MTT-S members and has the ability to delegate tasks as necessary to achieve the purposes of the society chapter. The MTT-S Vice Chair serves as the junior executive leader of Purdue IEEE MTT-S and shall set forth a technical direction for the academic year, ensure that events are planned for successful outcome, and otherwise assist the MTT-S Chair with his or her duties.

The MTT-S Sponsorship Coordinator shall be an industrial point of contact for the society chapter and will work with companies, faculty, and campus entities to acquire funding for events and hold seminars of interest to MTT-S members. The MTT-S Workshop Coordinator will create plans for extended learning opportunities for MTT-S members that explore topics in depth or impart valuable skills to those in attendance. The MTT-S Event Coordinator is tasked with planning, marketing, and executing social activities that build personal relationships between members and facilitate mentoring in the specialized field of high-frequency technology.

The Chapter Faculty Advisor for Purdue IEEE MTT-S shall act as a non-voting member of the leadership. He or she shall guide the leadership to best serve the needs of the microwave technology community on campus and throughout IEEE, aid in the continuity of activities, and connect members to opportunities in academia or industry. The MTT-S Chair is responsible for informing the Chapter Faculty Advisor and delivering needed reports to higher levels of IEEE MTT-S of society chapter activities.


\article{Procedure for Decisions}
\label{art:decide}

The ideas and opinions of the collective members of Purdue IEEE MTT-S should guide all decisions within the committee. As the leader who balances the needs of the members, the MTT-S Chair shall have the authority to make decisions on behalf of the entire society chapter. However, any other leader in Purdue IEEE MTT-S may call for a vote among the leadership on the issue in question. A simple majority of Purdue IEEE MTT-S leaders voting in the manner decided by the MTT-S Chair will determine what the decision shall be if it differs from the original decision from the MTT-S Chair. There is no minimum on leaders present for a quorum. Proxy voting is disallowed.

The MTT-S Chair shall have the power to decide access to committee resources in a manner consistent with these bylaws. Workspace granted specifically to Purdue IEEE MTT-S by faculty or staff shall be in the stewardship of the MTT-S Chair and Chapter Faculty Advisor. Additionally, the MTT-S Chair will have full control over the budgeting of all expenses and approval of all purchases. These privileges may be extended to others by order of the MTT-S Chair. The IEEE Treasurer will only reimburse purchases made with approval from Purdue IEEE MTT-S.
\todo[inline,backgroundcolor=red!25]{MRH: How is the MTT-S Chair going to ensure the MTT-S budget for the fiscal year is delivered to the IEEE Treasurer within the first week of the academic year in accordance with Article X, Section 4 of the Constitution of IEEE?}


\article{Elections to and Departures from Leadership}
\label{art:electdepart}

The MTT-S Chair, MTT-S Vice Chair, MTT-S Sponsorship Coordinator, MTT-S Workshop Coordinator, and MTT-S Event Coordinator shall be elected among the voting Purdue IEEE MTT-S members. The MTT-S Chair and other elected leaders should enroll as members of IEEE and IEEE Microwave Theory and Techniques Society additionally for the duration of service. Elections should occur at least once per academic year at least one day prior to the elections of the Purdue IEEE Student Branch. The term of office starts on \datetermstart{}.

The MTT-S Chair shall decide a teller for the elections. The teller shall have the authority over vote counting and the ability to decide which votes on a ballot are valid. The specifics of nominations, voting process, and tie breaking shall be given in the Constitution of IEEE.

A leader of Purdue IEEE MTT-S may resign their role at any time. Twelve different MTT-S members may sign a petition requesting the initiation of the recall process for any specific leader. The recall is successful when a simple majority of the Purdue IEEE MTT-S leaders vote in favor of recalling the leader. Vacancies in office shall be handled in the manner described below.

The MTT-S leadership shall normally be elected by the method outlined in the Constitution of IEEE. The MTT-S Chair and MTT-S Vice Chair positions should be occupied at all times by different elected individuals. In the event of a vacancy of the MTT-S Vice Chair prior to elections, the MTT-S Chair shall choose a successor who holds the office until an election may be held. In the event of a vacancy of the MTT-S Chair prior to elections, the MTT-S Vice Chair shall immediately assume the role of MTT-S Chair and plan for a speedy election of the leadership within the month. Should both the MTT-S Chair and MTT-S Vice Chair simultaneously be unoccupied, then the IEEE Vice President shall work with the remaining leadership to elect those roles to ensure the livelihood of the committee. Extended vacancies of the MTT-S Sponsorship Coordinator, MTT-S Workshop Coordinator, or MTT-S Event Coordinator are allowed provided that the remaining leadership performs the duties of those positions.


\article{Amendments to Bylaws}
\label{art:amend}

Any amendments to these bylaws shall be proposed by the MTT-S Chair. The bylaws shall pass with a two-thirds vote of the leadership. Passed bylaws shall take effect when recognized by the Executive Committee of Purdue IEEE Student Branch as described in the Constitution of IEEE.


\article{Society Dues}
\label{art:dues}

Purdue IEEE MTT-S chooses to forego dues specific to this local society chapter beyond those required by the student branch as a whole. A reversal of this position requires an amendment to these bylaws.


\article{Subcommittees}
\label{art:subcommittee}

The MTT-S Chair has the right to form subcommittees under a named director in Purdue IEEE MTT-S to accomplish mutually agreed upon goals. The continued existence of the subcommittee and the position of the director are at the discretion of the MTT-S Chair. All operations of the subcommittee are under the purview of the MTT-S Chair and MTT-S Vice Chair. A majority vote of the existing MTT-S leadership is needed to add any subcommittee director to MTT-S leadership.


\todo[inline,backgroundcolor=red!25]{MRH: How is MTT-S going to ensure that there are enough representatives sent to the Cornerstones committees if additional restrictions are imposed in the future?}

\vspace{12pt}
\hrule

\textit{Effective: February 24\textsuperscript{th}, 2017}


\setcounter{tocdepth}{1}
\listoftodos % Comment out when not editing


\end{document}