% Preamble
\documentclass[12pt]{constitution}
\usepackage{mathpazo, graphicx}
\graphicspath{{figures/}}
\usepackage[draft]{todonotes} % Show comments in PDF
%\usepackage[disable]{todonotes} % Hide comments from PDF

% Metadata
\title{EV Racing Team Bylaws}
\author{Technical Committee of Purdue IEEE Student Branch}
\date{}
\newcommand{\datetermstart}{June 1\textsuperscript{st}} % Place term start date here
\newcommand{\dateannualreportsubmit}{May 1\textsuperscript{st}} % Place IEEE Annual Student Report submission deadline here
\newcommand{\datefiscalstart}{August 1\textsuperscript{st}} % Place fiscal year start date here
\newcommand{\dateelectionsheld}{May 1\textsuperscript{st}} % Place last day elections may be held here


\begin{document}
% Title Option #1: Title Page and Table of Contents Separate
%\maketitle
%\setcounter{tocdepth}{0} % Change to 0 for articles only or 1 for articles and sections
%\tableofcontents
%\newpage

% Title Option #2: Title at Start of Table of Contents
\titlecontentspage
\newpage


\article*{IEEE EV Racing}
\label{art:racing}

The IEEE Electric Vehicle (EV) Racing Team exists as a committee under the Purdue IEEE Student Branch. The mission of the IEEE EV Racing Team is to foster technical and professional skills of its members by designing, manufacturing, and implementing electric vehicles to compete in the EV Grand Prix Competition and or any other Racing events that the members decide relevant.


\article{Purpose}
\label{art:purpose}

IEEE EV Racing Team focuses on designing and manufacturing electric vehicles. The types of vehicles range from electric skateboards to electric dune buggies, but the committee primarily focuses on an electric go-kart. Team members will apply their interest in electrical and vehicular technology and classroom concepts to real-world situations.

\article{Membership and Leadership}
\label{art:memlead}

The IEEE EV Racing team, at minimum, will consist of a Racing Captain (who shall be IEEE committee chair), Sponsorship Coordinator, Workshop Coordinator, Event Coordinator. The Racing Captain may add any other leadership role he or she deems necessary. The Racing Captain is responsible for organizing the team, setting a technical direction, and making all decisions that best achieves the purpose of the IEEE EV Racing Team. The captain will be the central point of contact between the team and the rest of Purdue IEEE Student Branch. The Sponsorship Coordinator, Workshop Coordinator, and Event Coordinator are responsible for attending the IEEE Professional Committee, IEEE Learning Committee, and IEEE Social Committee meetings, respectively, in addition to completing any documentation necessary for internal committee communication.

In order to be an official member of IEEE and any other committee under the Purdue IEEE Student Branch, a person must pay the IEEE local dues and comply with other requirements of the IEEE Executive Committee at the minimum. Infrequent attendance, little or no contributions made, disruptive behavior, or lack of IEEE dues payment may be grounds to not be recognized as a Racing member.


\article{Decisions}
\label{art:decide}

While decisions should usually be made with the advice and consent of the team, the Racing Captain shall have the ability to make or delegate all final team decisions. However, if there is a disagreement, a unanimous vote of leadership shall result in a vote over the decision being taken by the whole team. A majority of voting team members will override any captain?s decision. The captain shall have the power to decide access to team resources in a manner consistent with these bylaws. Additionally, the captain will be responsible for the approval of team purchases. This process may be delegated by the captain as needed. Only with approval will a purchase be eligible for reimbursement by the IEEE Treasurer. The Racing Captain will be responsible for delivering the Racing budget for the season to the IEEE Treasurer within the first week of the academic year. While members of the team may be removed or demoted by the captain, the captain may lose his or her position only in the same voting procedure listed above.

In the event of a temporary captain vacancy, the Racing Captain may appoint an interim replacement in his or her absence. If the vacancy is longer than one month, an election should occur as soon as reasonably possible in the same way as defined below. The Racing Captain may appoint an acting captain for any length of time during his or her captaincy with a majority approval of the leadership.

\article{Elections}
\label{art:elect}

By April 27 of the current year, there shall be a vote for Racing Captain. A season shall last no longer than 16 months. A captain must have been a member of the team for at least four months. All current members of the team at the time of the election shall have one vote. Eligibility for voting shall be decided by the captain with regards to the aforementioned bylaws. A majority of voting team members shall elect the captain. In the event no majority exists, the member with the fewest votes shall be removed from the ballot and another vote will occur. If there is still no majority when there are only two members left, the previous captain shall decide the new captain. Only when fewer than four members are present at elections will the captain at the time make the decision for next captain that is in the best interest for IEEE EV Racing and Purdue IEEE as a whole.


\article{Bylaws}
\label{art:bylaw}

Any amendments to the bylaws are to be proposed in writing and will be subject to approval by the Racing Captain. The bylaws shall be implemented with a majority of voting team members.

\vspace{12pt}
\hrule

\textit{Effective: March 1\textsuperscript{st}, 2017}


\setcounter{tocdepth}{1}
\listoftodos % Comment out when not editing


\end{document}