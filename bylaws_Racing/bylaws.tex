% Preamble
\documentclass[12pt]{constitution}
\usepackage{mathpazo, graphicx}
\graphicspath{{figures/}}
\usepackage[draft]{todonotes} % Show comments in PDF
%\usepackage[disable]{todonotes} % Hide comments from PDF

% Metadata
\title{Racing Team Bylaws}
\author{Technical Committee of Purdue IEEE Student Branch}
\date{}
\newcommand{\datetermstart}{June 1\textsuperscript{st}} % Place term start date here
\newcommand{\dateannualreportsubmit}{May 1\textsuperscript{st}} % Place IEEE Annual Student Report submission deadline here
\newcommand{\datefiscalstart}{August 1\textsuperscript{st}} % Place fiscal year start date here
\newcommand{\dateelectionsheld}{May 1\textsuperscript{st}} % Place last day elections may be held here


\begin{document}
% Title Option #1: Title Page and Table of Contents Separate
%\maketitle
%\setcounter{tocdepth}{0} % Change to 0 for articles only or 1 for articles and sections
%\tableofcontents
%\newpage

% Title Option #2: Title at Start of Table of Contents
\titlecontentspage
\newpage


\article*{Preamble}
\label{art:preamble}

The IEEE Racing Team exists as a committee under the Purdue IEEE Student Branch. The mission of the IEEE Racing Team is to foster technical and professional skills of its members by designing, manufacturing, and testing electric vehicles in order to compete in the evGrandPrix competition and any other motorsports events that the members decide are relevant. All Racing team members are required to follow the rules established in these bylaws in addition to the Constitution of IEEE.


\article{Membership and Leadership}
\label{art:memlead}

The IEEE Racing team, at minimum, will consist of a captain (who shall be IEEE technical committee chair and vote within the Executive Committee of Purdue IEEE Student Branch), mechanical lead, electrical lead, and controls lead. The captain may add any other leadership role he or she deems necessary without a change or amendment to the bylaws, and the additional leadership roles (besides the captain, mechanical lead, electrical lead, control lead) can also be subtracted by the captain without a change or amendment to the bylaws. The captain is responsible for organizing the team, setting a technical direction, and making all decisions that best achieve the purpose of the IEEE Racing Team. The captain appoints and assigns work to the technical leads named above (i.e., mechanical lead, electrical lead, and controls lead) to ensure smooth operation of the technical teams and timely completion of all tasks.

The captain has the authority to delegate his or her responsibilities to a vice captain should the captain deem it necessary. Duties that must remain with the captain and not the vice captain nor any other Racing team member include final budget decisions, unlimited reimbursement approval power, changes to composition or organization of the team roster, and selection of members to attend the evGrandPrix competition with team funds. Other positions besides the technical leads may be appointed and removed at the captain’s leisure.

Attendance at a majority of team meetings and continuous participation with meaningful contributions throughout the design, construction, and testing phases of IEEE Racing are expected of all members. Attendance at a majority of technical team meetings to which the respective member belongs is expected. All members must attend at least 50\% of all meetings and may not miss more than three (3) meetings in a row unless explicit written permission is given by the captain. Additionally, in order to be an official member of IEEE Racing Team, a member must meet the dues requirement for Purdue IEEE Student Branch. Infrequent attendance, little or no contributions made, disruptive behavior, or lack of IEEE dues payment may be cited as reasons to not be recognized as a Racing member.

\article{Decisions}
\label{art:decide}

While decisions should usually be made with the advice and consent of the team, the captain shall have the ability to make or delegate all final team decisions. However, if there is a disagreement, a unanimous vote of leadership shall result in a vote over the decision being taken by the whole team (does not include the captain). A majority of voting team members will override any captain's decision.

The captain shall have the power to decide access to team resources in a manner consistent with these bylaws. Additionally, the captain will be responsible for the approval of team purchases. This process may be delegated by the captain as needed. Only with approval will a purchase be eligible for reimbursement by the IEEE Treasurer. The captain will be responsible for delivering the IEEE Racing budget for the season to the IEEE Treasurer within the first week of the academic year.

The captain shall ensure that all members are trained in their respective technical fields in alignment with
the goals of IEEE Learning Committee at the start of the academic year.

While members of the team may be removed or demoted by the captain, the captain may lose his or her position in the same voting procedure listed in Article III, Paragraph I.

In the event of a temporary captain vacancy lasting less than one month, the captain may appoint an interim replacement with the full powers of the captain in his or her absence. The acting captain shall retain the full powers of captain until the return of the original captain, at which point, the full powers of captain return to the original captain. In all other circumstances, an election for a new captain should be held as soon as reasonably possible in the way defined in Article III.

\article{Elections}
\label{art:elect}

By \dateelectionsheld of a given year, there shall be a vote for captain, with the elected captain assuming the position on June 1st and serving until May 31\textsuperscript{st} of the following year. The newly elected captain should participate in the IEEE Executive Committee for the class taking office after the season ends.

A captain must have been a member of the team for at least four months and satisfy all other eligibility requirements imposed by the Constitution of IEEE. In case where no members have been in the committee for more than 4 months, the captain will be appointed by the IEEE Executive Committee according to the IEEE Constitution. All current members of the team at the time of the election shall have one vote. Eligibility for voting shall be decided by the captain with regards to the aforementioned bylaws. A majority of voting team members shall elect the captain. In the event no majority exists, the candidate with the fewest votes shall be removed from the ballot and another vote will occur. If there is still no majority when there are only two members left, the previous captain shall decide the new captain. Only when fewer than four members are present at elections will the captain at the time forgo a vote and make the decision for the next captain that is in the best interest for IEEE Racing Team and Purdue IEEE Student Branch as a whole. 

\article{Bylaws}
\label{art:bylaw}

Any amendments to the bylaws shall be proposed by any member and be brought to a vote only upon approval by the captain. The bylaws shall pass with a majority of voting team members or by a unanimous team lead vote.

\vspace{12pt}
\hrule

\textit{Effective: September 30\textsuperscript{th}, 2022}


\setcounter{tocdepth}{1}
%\listoftodos % Comment out when not editing


\end{document}
