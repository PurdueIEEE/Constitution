% Preamble
\documentclass[12pt]{constitution}
\usepackage{mathpazo, graphicx}
\graphicspath{{figures/}}

% Metadata
\title{Purdue Aerial Robotics Team Committee Bylaws}
\author{Technical Committee of Purdue IEEE Student Branch}
\date{}
\newcommand{\datetermstart}{June 1\textsuperscript{st}} % Place term start date here
\newcommand{\dateannualreportsubmit}{May 1\textsuperscript{st}} % Place IEEE Annual Student Report submission deadline here
\newcommand{\datefiscalstart}{August 1\textsuperscript{st}} % Place fiscal year start date here
\newcommand{\dateelectionsheld}{May 1\textsuperscript{st}} % Place last day elections may be held here


\begin{document}
% Title Option #1: Title Page and Table of Contents Separate
%\maketitle
%\setcounter{tocdepth}{0} % Change to 0 for articles only or 1 for articles and sections
%\tableofcontents
%\newpage

% Title Option #2: Title at Start of Table of Contents
\titlecontentspage
\newpage

\article{Name and Purpose}
\label{art:nampup}
The Purdue Aerial Robotics Team (PART) exists as a committee under the Purdue IEEE Student Branch. The mission of the Purdue Aerial Robotics Team is to foster technical and professional skills of its members by designing, manufacturing, and testing an autonomous unmanned aerial vehicle to compete in the AUVSI-SUAS Competition. All Team members are required to follow all rules established by these bylaws and the Constitution of IEEE.


\article{Membership and Leadership}
\label{art:memberlead}

PART, at a minimum, shall consist of a Purdue IEEE Committee Chair (henceforth referred to as captain), aeromechanical lead, electrical/software lead, and financial lead. The captain may add or remove leadership roles as he or she deems necessary. The captain is responsible for organizing the team, setting a technical direction, and making all decisions to best fulfil the mission of PART. The captain will be the central point of contact between PART and the rest of the Purdue IEEE Student Branch. The captain appoints and assigns tasks to the technical and financial leadership leads named above at his or her discretion.

The captain has the authority to delegate some of his or her responsibilities to a vice captain should the captain deem it necessary. Duties that must remain with the captain and can not be performed by the vice captain nor any other member include final budget decisions, changes to composition or organization of the team roster, and selection of members to attend the AUVSI-SUAS competition. If the captain deems it necessary, the captain may create project groups at any time. Each project group has a project group head appointed by the captain with the duties to fulfil the project group task. Each project group head is responsible for providing timely reports to the team captain and resolving minor disputes between the project group members. Project group heads are not considered as official leadership positions. The captain may add or subtract or change membership of the project group at any time.

To be considered a PART member, one must attend a majority of team meetings and provide continuous meaningful contributions throughout the design, manufacturing, and testing activities. Attendance at a majority of technical team meetings to which the member belongs is expected. Additionally, all members are required to be a member of the Purdue IEEE Student Branch by, including but not limited to, paying all local IEEE dues. Infrequent attendance, infrequent contributions, disruptive behavior, or lack of Purdue IEEE student membership may be grounds for lack of PART membership. All PART members are considered voting members. Graduate students pursuing a masters, doctoral, or post-doctoral degree may only be considered PART members at the discretion of the captain.


\article{Decisions}
\label{art:decisions}

While decisions should usually be made with the advice and consent of the team, the captain shall have the ability to make all final team decisions. However, if there is a disagreement with the leadership committee, a unanimous vote of the leads shall result in a vote over the decision being taken by the whole team. A majority of voting team members will override any captain’s decision.

The captain shall have the power to decide access to the team resources in a manner consistent with these bylaws. Additionally, the captain will be responsible for the approval of team purchases. This process may be delegated by the captain as needed. Only with approval will a purchase be eligible for reimbursement by the IEEE Treasurer. The captain is responsible for delivering the PART budget for the season to the IEEE Treasurer within the first week of the academic year in accordance with the Constitution of IEEE.

While members of the team may be removed or demoted by the captain, the captain may lose his or her position in the same voting procedure listed in Article III.

In the event of a temporary captain vacancy less than one month, the captain may appoint a temporary replacement with full powers of the captain in his or her absence. In all other circumstances, an election for an acting captain with full powers of captain should be held as soon as reasonably possible in the manner defined in Article III. The acting captain shall retain the full power of captain until the return of the original captain, at which point the full powers of captain return to the original captain.


\article{Elections}
\label{art:elections}

Before each new season, there shall be a vote for captain. A season shall consist of no more than one international competition. The election for captain should occur at least one day before the Purdue IEEE Student Branch elections; however, the current captain may request a deferral of the vote to the elected officers of IEEE. The elected captain should participate in IEEE Executive Committee meetings for the IEEE election term after the season ends. The current captain shall remain in power for the remainder of the season.

A captain must have been a member of the team for at least four months and satisfy all other eligibility requirements imposed by the Constitution of IEEE. All current members of the team at the time of the election shall have one vote. Eligibility for voting shall be decided by the captain with regards to the aforementioned bylaws. A majority of voting team members shall elect the captain. In the event no majority exists, the candidate with the fewest votes shall be removed from the ballot and another vote will occur. If there is still no majority among the voting members when there are only two candidates left, the leadership leads shall hold a vote with only leadership leads voting. If still no majority exists among the vote by the leadership leads, the previous captain shall decide the new captain.


\article{Bylaws}
\label{art:procedure}
Any amendment to the bylaws shall be proposed by any member and be brought to a vote only upon approval by the captain. The bylaws shall pass with a majority of voting members or by a unanimous leadership lead vote.


\vspace{12pt}
\hrule

\textit{Effective: August 25\textsuperscript{th}, 2021}


\setcounter{tocdepth}{1}


\end{document}
