% Preamble
\documentclass[12pt]{constitution}
\usepackage{mathpazo, graphicx}
\graphicspath{{figures/}}
\usepackage[draft]{todonotes} % Show comments in PDF
%\usepackage[disable]{todonotes} % Hide comments from PDF

% Metadata
\title{Aerial Robotics Team (PARTIEEE) Bylaws}
\author{Technical Committee of Purdue IEEE Student Branch}
\date{}
\newcommand{\datetermstart}{June 1\textsuperscript{st}} % Place term start date here
\newcommand{\dateannualreportsubmit}{May 1\textsuperscript{st}} % Place IEEE Annual Student Report submission deadline here
\newcommand{\datefiscalstart}{August 1\textsuperscript{st}} % Place fiscal year start date here
\newcommand{\dateelectionsheld}{May 1\textsuperscript{st}} % Place last day elections may be held here


\begin{document}
% Title Option #1: Title Page and Table of Contents Separate
%\maketitle
%\setcounter{tocdepth}{0} % Change to 0 for articles only or 1 for articles and sections
%\tableofcontents
%\newpage

% Title Option #2: Title at Start of Table of Contents
\titlecontentspage
\newpage


\article*{Preamble}
\label{art:preamble}

The Purdue Aerial Robotics Team IEEE (PARTIEEE) exists as a committee under the Purdue IEEE Student Branch. The mission of PARTIEEE is for members to deliver a custom UAS solution to the AUVSI SUAS Competition while gaining technical skills, professional growth, and interpersonal relationships. All PARTIEEE members are required to follow the rules established in these bylaws in addition to the Constitution of IEEE.


\article{Structure and Membership}
\label{art:structmem}

PARTIEEE shall, at minimum, be composed of a captain (who shall be the technical committee chair and vote within the Executive Committee of Purdue IEEE Student Branch), secretary, aeromech subteam lead, electrical subteam lead, and software subteam lead. Other, non-subteam lead positions may be appointed and removed at the suggestion and popular vote of the captain, secretary, and subteam leads. Social, professional, and learning delegates may be appointed and removed at the captain?s leisure to aid in those responsibilities.

Attendance at a majority of team meetings and continuous participation with meaningful contributions throughout the design, construction, testing, and documentation phases of the UAS are expected of all members. Additionally, attendance at a majority of technical subteam meetings to which the respective member belongs is expected. Infrequent attendance, low number or quality of contributions made, disruptive behavior, or lack of IEEE dues payment may be cited as reasons to not be recognized as a member, selected to represent the team at outreach events, or allowed to attend competitions.


\article{Decisions}
\label{art:decide}

While decisions should usually be made with the advice and consent of the team, the captain shall have the ability to make or delegate all final team decisions. However, if there is a disagreement, a unanimous vote of subteam leads and secretary shall result in a vote over the decision being taken by the whole team. A majority of voting team members will override any captain?s decision.

The captain shall have the power to decide access to team resources in a manner consistent with these bylaws. Additionally, the captain will be responsible for the approval of team purchases. This process may be delegated by the captain as needed. Only with approval will a purchase be eligible for reimbursement by the IEEE Treasurer. The captain and secretary are jointly responsible for finalizing the budget which the captain will deliver to the IEEE Treasurer within the first week of the academic year. While members of the team may be removed or demoted by the captain, the captain may lose their position only in the same voting procedure listed above.

In the event of a temporary captain vacancy, the captain may appoint a temporary replacement in his or her absence. If the vacancy is longer than one month, an election should occur as soon as reasonably possible in the same way as defined below. The captain may appoint an acting captain for any length of time during his or her captaincy with a majority approval of the subteam leads and secretary. Should the secretary or any subteam leads knowingly leave a vacancy, they are to appoint their replacement to be voted upon by the other subteam leads and/or secretary. Should a subteam lead or secretary leave unexpectedly, the member to be voted upon is to be appointed by and voted upon by the other subteam leads and/or secretary.

\article{Elections}
\label{art:elect}

\todo[inline,backgroundcolor=red!25]{MRH: According to Article IV, Section 1 of the Constitution of IEEE; the technical committee chair must be decided by election time. How are you going to ensure compliance within your bylaws?}
Before the beginning of each new season, there shall be a vote for captain\todo[backgroundcolor=red!25]{MRH: and secretary}. A season shall consist of no more than one international competition. The captain and secretary \todo[backgroundcolor=red!25]{MRH: will be}selected by a vote between the current captain, secretary, and subteam leads upon the end of the previous captain or secretary?s season. Subteam leads are to be selected by the previous subteam lead before their term ends, and an application and interview process are to be used in this decision.


\article{Bylaws}
\label{art:bylaw}

Any amendments to the bylaws shall be proposed to and subject to approval by the captain. The bylaws shall pass with a majority of voting team members or by a unanimous subteam lead vote.

\todo[inline,backgroundcolor=red!25]{MRH: How is PARTIEEE going to ensure that there are representatives sent to the Cornerstones committees?}

\vspace{12pt}
\hrule

\textit{Effective: February 28\textsuperscript{th}, 2017}


\setcounter{tocdepth}{1}
\listoftodos % Comment out when not editing


\end{document}