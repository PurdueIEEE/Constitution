% Preamble
\documentclass[12pt]{constitution}
\usepackage{mathpazo, graphicx}
\graphicspath{{figures/}}
\usepackage[draft]{todonotes} % Show comments in PDF
%\usepackage[disable]{todonotes} % Hide comments from PDF

% Metadata
\title{Computer Society Chapter Bylaws}
\author{Technical Committee of Purdue IEEE Student Branch}
\date{}
\newcommand{\datetermstart}{June 1\textsuperscript{st}} % Place term start date here
\newcommand{\dateannualreportsubmit}{May 1\textsuperscript{st}} % Place IEEE Annual Student Report submission deadline here
\newcommand{\datefiscalstart}{August 1\textsuperscript{st}} % Place fiscal year start date here
\newcommand{\dateelectionsheld}{May 1\textsuperscript{st}} % Place last day elections may be held here


\begin{document}
% Title Option #1: Title Page and Table of Contents Separate
%\maketitle
%\setcounter{tocdepth}{0} % Change to 0 for articles only or 1 for articles and sections
%\tableofcontents
%\newpage

% Title Option #2: Title at Start of Table of Contents
\titlecontentspage
\newpage


\article{Name and Purpose}
\label{art:namepurp}

Purdue IEEE Computer Society exists as a Student Branch Chapter under the Purdue IEEE Student Branch. In the spirit of the purpose of the organization, the mission of IEEE Computer Society is to advance the field of computer science and technology. All Computer Society members are required to follow the rules established in these bylaws in addition to the Constitution of IEEE.


\article{Membership}
\label{art:member}

The members of Purdue IEEE Computer Society are those from the Purdue IEEE Student Branch who voluntarily associate themselves with this committee. Active members are those who attend at least three events specific to Computer Society each academic term in addition to prompt payment of all local IEEE dues and other requirements for good standing in Purdue IEEE Student Branch. The undergraduate and graduate student members that are active members of Purdue IEEE Computer Society are entitled to the full rights and voting privileges of members. Membership in IEEE Computer Society at the international level is separate from local membership and alone is not sufficient to be considered active in the local chapter.

The Chapter Faculty Advisor will be a member of Purdue IEEE Computer Society that is part of the faculty of Purdue University active in both IEEE and IEEE Computer Society in particular. The leadership of Purdue IEEE Computer Society has the right to extend honorary membership to any person. The Chapter Faculty Advisor and all honorary members lack the ability to vote.


\article{Leadership}
\label{art:leader}

The Purdue IEEE Computer Society leadership shall consist of the following voting individuals:
\begin{itemize}
    \item Computer Society Chair
    \item Computer Society Vice Chair
    \item Computer Society Sponsorship Coordinator
    \item Computer Society Workshop Coordinator
    \item Computer Society Event Coordinator
    \item Approved subcommittee directors
\end{itemize}

The Computer Society Chair alone will exercise the ability to vote within the Executive Committee of Purdue IEEE Student Branch. The Computer Society Sponsorship Coordinator, Computer Society Workshop Coordinator, and Computer Society Event Coordinator shall be the \textit{ex officio} Professional Committee representative, Learning Committee representative, and Social Committee representative respectively to the Cornerstones committees of Purdue IEEE Student Branch on behalf of Computer Society members.

The Computer Society Chair will preside over all meetings of Purdue IEEE Computer Society and transact all business necessary to address the administrative needs of the society chapter. He or she shall strive to meet the needs of other Computer Society members and has the ability to delegate tasks as necessary to achieve the purposes of the society chapter. The Computer Society Vice Chair serves as the junior executive leader of Purdue IEEE Computer Society and shall set forth a technical direction for the academic year, ensure that events are planned for successful outcome, and otherwise assist the Computer Society Chair with his or her duties. The Computer Society Chair has the final say in every decision that corresponds to IEEE Computer Society Chapter.

The Computer Society Sponsorship Coordinator shall be an industrial point of contact for the society chapter and will work with companies, faculty, and campus entities to acquire funding for events and hold seminars of interest to Computer Society members. The Computer Society Workshop Coordinator will create plans for extended learning opportunities for Computer Society members that explore topics in depth or impart valuable skills to those in attendance. The Computer Society Event Coordinator is tasked with planning, marketing, and executing social activities that build personal relationships between members and facilitate mentoring in the specialized field of computer technology.

The Chapter Faculty Advisor for Purdue IEEE Computer Society shall act as a non-voting member of the leadership. He or she shall guide the leadership to best serve the needs of the Computer Society community on campus and throughout IEEE, aid in the continuity of activities, and connect members to opportunities in academia or industry. The Computer Society Chair is responsible for informing the Chapter Faculty Advisor and delivering needed reports to higher levels of IEEE Computer Society of society chapter activities.


\article{Procedure for Decisions}
\label{art:decide}

The ideas and opinions of the collective members of Purdue IEEE Computer Society should guide all decisions within the committee. As the leader who balances the needs of the members, the Computer Society Chair shall have the authority to make decisions on behalf of the entire society chapter. However, any other leader in Purdue IEEE Computer Society may call for a vote among the leadership on the issue in question. A simple majority of Purdue IEEE Computer Society leaders voting in the manner decided by the Computer Society Chair will determine what the decision shall be if it differs from the original decision from the Computer Society Chair. There is no minimum on leaders present for a quorum. Proxy voting is disallowed.

The Computer Society Chair shall have the power to decide access to committee resources in a manner consistent with these bylaws. Workspace granted specifically to Purdue IEEE Computer Society by faculty or staff shall be in the stewardship of the Computer Society Chair and Chapter Faculty Advisor. Additionally, the Computer Society Chair will have full control over the budgeting of all expenses and approval of all purchases. These privileges may be extended to others by order of the Computer Society Chair. The IEEE Treasurer will only reimburse purchases made with approval from Purdue IEEE Computer Society.
\todo[inline,backgroundcolor=red!25]{MRH: How is the Computer Society Chair going to ensure the Computer Society budget for the fiscal year is delivered to the IEEE Treasurer within the first week of the academic year in accordance with Article X, Section 4 of the Constitution of IEEE?}


\article{Elections to and Departures from Leadership}
\label{art:electdepart}

The Computer Society Chair, Computer Society Vice Chair, Computer Society Sponsorship Coordinator, Computer Society Workshop Coordinator, and Computer Society Event Coordinator shall be elected among the voting Purdue IEEE Computer Society members. The Computer Society Chair and other elected leaders should enroll as members of IEEE and IEEE Computer Society additionally for the duration of service. Elections should occur at least once per academic year at least one day prior to the elections of the Purdue IEEE Student Branch. The term of office starts on \datetermstart{}.

The Computer Society Chair shall decide a teller for the elections. The teller shall have the authority over vote counting and the ability to decide which votes on a ballot are valid. The specifics of nominations, voting process, and tie breaking shall be given in the Constitution of IEEE.

The Computer Society leadership shall normally be elected by the method outlined in the Constitution of IEEE. The Computer Society Chair, however all the other leadership positions will be filled as per nominations by The Computer Society Chair. In the event of a vacancy of the Computer Society Vice Chair prior to elections, the Computer Society Chair shall choose a successor who holds the office until an election may be held. In the event of a vacancy of the Computer Society Chair prior to elections, the Computer Society Vice Chair shall immediately assume the role of Computer Society Chair and plan for a speedy election of the leadership within the month. Should both the Computer Society Chair and Computer Society Vice Chair simultaneously be unoccupied, then the IEEE Vice President shall work with the remaining leadership to elect those roles to ensure the livelihood of the committee. Extended vacancies of the Computer Society Sponsorship Coordinator, Computer Society Workshop Coordinator, or Computer Society Event Coordinator are allowed provided that the remaining leadership performs the duties of those positions. The Computer Society Chair can at will resign from the position of the Chair.


\article{Amendments to Bylaws}
\label{art:amend}

Any amendments to these bylaws shall be proposed by the Computer Society Chair. The bylaws shall pass with a two-thirds vote of the leadership. Passed bylaws shall take effect when recognized by the Executive Committee of Purdue IEEE Student Branch as described in the Constitution of IEEE.


\article{Society Dues}
\label{art:dues}

Purdue IEEE Computer Society chooses to forego dues specific to this local society chapter beyond those required by the student branch as a whole. A reversal of this position requires an amendment to these bylaws.


\article{Subcommittees}
\label{art:subcommittee}

The Computer Society Chair has the right to form subcommittees under a named director in Purdue IEEE Computer Society to accomplish mutually agreed upon goals. The continued existence of the subcommittee and the position of the director are at the discretion of the Computer Society Chair. All operations of the subcommittee are under the purview of the Computer Society Chair and Computer Society Vice Chair. A majority vote of the existing Computer Society leadership is needed to add any subcommittee director to Computer Society leadership.


\todo[inline,backgroundcolor=red!25]{MRH: How is Computer Society going to ensure that there are enough representatives sent to the Cornerstones committees if additional restrictions are imposed in the future?}

\vspace{12pt}
\hrule

\textit{Effective: February 24\textsuperscript{th}, 2017}


\setcounter{tocdepth}{1}
\listoftodos % Comment out when not editing


\end{document}